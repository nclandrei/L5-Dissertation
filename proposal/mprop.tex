\documentclass{mprop}
\usepackage{graphicx}

% alternative font if you prefer
%\usepackage{times}

% for alternative page numbering use the following package
% and see documentation for commands
%\usepackage{fancyheadings}


% other potentially useful packages
%\uspackage{amssymb,amsmath}
%\usepackage{url}
%\usepackage{fancyvrb}
%\usepackage[final]{pdfpages}

\begin{document}

%%%%%%%%%%%%%%%%%%%%%%%%%%%%%%%%%%%%%%%%%%%%%%%%%%%%%%%%%%%%%%%%%%%
\title{Software Ticket Quality}
\author{Andrei-Mihai Nicolae}
\date{December 18th 2017}
\maketitle
%%%%%%%%%%%%%%%%%%%%%%%%%%%%%%%%%%%%%%%%%%%%%%%%%%%%%%%%%%%%%%%%%%%

%%%%%%%%%%%%%%%%%%%%%%%%%%%%%%%%%%%%%%%%%%%%%%%%%%%%%%%%%%%%%%%%%%%
\tableofcontents
\newpage
%%%%%%%%%%%%%%%%%%%%%%%%%%%%%%%%%%%%%%%%%%%%%%%%%%%%%%%%%%%%%%%%%%%

%%%%%%%%%%%%%%%%%%%%%%%%%%%%%%%%%%%%%%%%%%%%%%%%%%%%%%%%%%%%%%%%%%%
\section{Introduction}\label{intro}
briefly explain the context of the project problem

% \textbf{}\cite{}:
% \begin{itemize}
%   \item \textbf{What were the goals?}
%   \item \textbf{What was the method?}
%   \item \textbf{What did they learn?}
%   \item \textbf{Relevance to our work}:
% \end{itemize}

\subsection{A subsection}
Please note your proposal need not follow the included section headings; this is only a suggested structure. Also add subsections etc as required

%%%%%%%%%%%%%%%%%%%%%%%%%%%%%%%%%%%%%%%%%%%%%%%%%%%%%%%%%%%%%%%%%%%
\section{Statement of Problem}

Clearly state the problem to be addressed in your forthcoming project.
Explain why it would be worthwhile to solve this problem.

%%%%%%%%%%%%%%%%%%%%%%%%%%%%%%%%%%%%%%%%%%%%%%%%%%%%%%%%%%%%%%%%%%%
\section{Background Survey}

\subsection{Introduction}

\subsection{What was covered and why}

\subsection{How papers were found; what terms were searched for + snowballing}

\subsection{Data Quality Metrics}

\textbf{Data Quality in Context}\cite{strong1997data}:
\begin{itemize}
  \item \textbf{What were the goals?}
    The main goal of the authors was to analyze data quality in different 
    organizations, observe data quality problem patterns as well as see 
    its implications.
  \item \textbf{What was the method?}
    They performed a qualitative analysis, examining 42 data quality projects 
    from 3 organizations. They collected data via interviews of data procedures,
    custodians, customers and managers. The authors analysed each project using
    the data quality dimensions as content analysis codes. 
  \item \textbf{What did they learn?}
    One thing that the authors learned is that representational data quality 
    dimensions are underlying causes of accessibility data quality problem 
    patterns. Also, they observed three underlying causes for users' complaints
    regarding data not supporting their tasks:
      \begin{itemize}
        \item incomplete data;
        \item inadequately defined or measured data;
        \item data that could not be appropriately aggregated.
      \end{itemize}
  \item \textbf{Relevance to our work}: this paper brings valuable insight into
    major data quality issues and patterns, information which we can apply in 
    our own research.
\end{itemize}

\textbf{Data Quality Assessment}\cite{pipino2002data}:
\begin{itemize}
  \item \textbf{What were the goals?}
    The main goal of the authors is to present ways of developing usable data
    quality metrics that organizations can implement in their own internal 
    processes.
  \item \textbf{What was the method?}
    They have researched previous methods of assessing data quality in information
    systems and they formulated general advice that can be directly implemented
    by organizations in their processes.
  \item \textbf{What did they learn?}
    They learned that there is no universal approach to assessing data quality
    as it heavily depends on the context where it is analysed.
  \item \textbf{Relevance to our work}: it shows how one can effectively measure
    data quality in information systems, thus we can apply it on the quality of
    the tickets we are analysing.
\end{itemize}

\textbf{Product perspective on total 
        Data Quality Assessment}\cite{wang1998product}:
\begin{itemize}
  \item \textbf{What were the goals?}
    
  \item \textbf{What was the method?}
    They created the TDQM methodology which is based on accumulated research 
    and extended 
  \item \textbf{What did they learn?}
  \item \textbf{Relevance to our work}:
\end{itemize}

\textbf{Knowing-Why about Data Processes and Data Quality}\cite{lee2003knowing}:
\begin{itemize}
  \item \textbf{What were the goals?}
    They tried to investigate whether knowing-why affects work performance and
    whether knowledge held by people in different roles affect the overall work
    performance. 
  \item \textbf{What was the method?}
    They had 6 companies that served as data collection points. In all of them,
    the research focused on the quality of their customer activity data, tuned
    for the specific domain the company was activating in. They held 2 sessions
    at each company: in the first session, the participants were presented with
    an overview of the research and a questionnaire, while during the second 
    session provided summary feedback on the questionnaire results and gave the
    participants the chance to provide comments and explanations.
  \item \textbf{What did they learn?}
    The major finding of the study was that there is vast complexity of knowledge
    at work in data production process. 
  \item \textbf{Relevance to our work}: an important research on how knowledge in
    different roles in a company can lead to decreased data quality overall. It
    will help us to analyse the software tickets we will mine better.
\end{itemize}

\textbf{Antecedents of Information and System Quality}
\cite{nelson2005antecedents}:
\begin{itemize}
  \item \textbf{What were the goals?}
    The authors had 2 main research objectives they wanted to aim for in this
    paper:
      \begin{itemize}
        \item identify a set of antecedents that both drive information and
          system quality, as well as define the nature of the IT artifact;
        \item explore data warehousing in general, especially analytical tools,
          predefined reports and ad hoc queries.
        \end{itemize}
  \item \textbf{What was the method?}
    They conducted a cross-functional survey to test the model the authors 
    proposed. They focused on user experience with a data warehouse. They sent
    email invites to the Data Warehouse Institute which distributed it further
    to organizations, among which 7 agreed to participate, yielding a total of
    465 completed surveys. 
  \item \textbf{What did they learn?}
    They learned that their selected features were indeed a good indicator of
    overall information and system quality. They also found out that accuracy is
    the dominant determinant across all three data warehouse technologies, 
    followed by completeness and format. Also, the results showed that more 
    attention needs to be given to differences across varying technologies. 
  \item \textbf{Relevance to our work}: this paper presents important findings
    regarding information and system quality in an IT environment. We can use it
    when analysing the quality of the tickets we will mine.
\end{itemize}

 \textbf{AIMQ}\cite{lee2002aimq}:
 \begin{itemize}
   \item \textbf{What were the goals?}
     The main challenge set by the authors was to develop an overall model
     with an accompanying assessment instrument for quantifying information
     quality.
   \item \textbf{What was the method?}
     They developed a methodology called AIM quality (hence the name AIMQ) which
     they applied at 5 different organizations. It has 3 main components:
      \begin{itemize}
        \item 2 $\times$ 2 model of what information quality means to managers;
        \item questionnaire for measuring information quality along the dimensions
          found in first step;
        \item two analysis techniques for interpreting the assessments captured 
          by the questionnaire.
      \end{itemize} 
   \item \textbf{What did they learn?}
      They conclude that the AIMQ methodology proved to be a practical information
      quality tool to organizations.
   \item \textbf{Relevance to our work}: the paper gave us valuable insight into 
     how to assess information quality.
\end{itemize}

\textbf{Evaluating IT Quality in E-Government 
        Environment}\cite{prybutok2008evaluating}:
\begin{itemize}
  \item \textbf{What were the goals?}
    The main goal of the paper is to identify if leadership and IT quality
    can lead to positive outcomes in an e-government environment.
  \item \textbf{What was the method?}
    They conducted a web survey to gather data and test their hypotheses. They
    assessed the City of Denton's e-government initiatives, including current 
    plans and implementations.
  \item \textbf{What did they learn?}
    They learned that the MBNQA leadership triad (leadership, strategic
    planning and customer/market focus) had a positive impact on the IT quality
    triad (information, system and service quality). Moreover, they found out
    that both leadership and IT quality improved the benefits.
  \item \textbf{Relevance to our work}: even though the environment presented
    in the paper was an e-government, it is useful to see how the organization 
    around a project as well as IT quality can lead to a better functioning of the 
    system. Thus, in our research, it can prove helpful in choosing the right 
    open source project to gather and analyse the data from.
\end{itemize}

\textbf{Organizational Impact of 
        Information Quality}\cite{gorla2010organizational}:
\begin{itemize}
  \item \textbf{What were the goals?}
    The authors had two main research questions they wanted to address:
      \begin{itemize}
        \item what individual/combined influences system quality, service quality,
          information quality have on organizational impact?
        \item what effect does system quality have on information quality?
      \end{itemize}
  \item \textbf{What was the method?}
    They conducted a construct measurement for the IS quality dimensions and 
    collected data through empirical testing.
  \item \textbf{What did they learn?}
    Firstly, they learned that service quality has the greatest impact 
    of all three quality constructs. Moreover, they found out that there is a 
    linkage between system and information quality, fact that previous research
    did not reveal. Lastly, their results indicate that the above-mentioned 
    quality dimensions have a significant positive influence on organizational
    impact either directly or indirectly.
  \item \textbf{Relevance to our work}: the paper can provide us insights into
    how valuable information quality is and what an impact it can have.
\end{itemize}

\textbf{Impact of Poor Data Quality}\cite{redman1998impact}:
\begin{itemize}
  \item \textbf{What were the goals?}
    The main purpose of the article is to provide some insights into how
    data quality can have a major economical impact on a company, as well
    as to raise awareness about this issue.
  \item \textbf{What was the method?}
    The author has analysed various large enterprises and their available data 
    to infer conclusions about the impact data quality has.
  \item \textbf{What did they learn?}
    There are three main issues across most enterprises:
      \begin{itemize}
        \item inaccurate data;
        \item inconsistencies across databases;
        \item unavailable data necessary for certain operations.
      \end{itemize}
    Moreover, the paper also presents a list of impacts poor data quality brings, among 
    which we can find:
      \begin{itemize}
        \item lowered customer satisfaction;
        \item increased cost;
        \item lowered employee satisfaction;
        \item it becomes more difficult to set strategy;
        \item poorer decision making.
      \end{itemize}
  \item \textbf{Relevance to our work}: this is a key paper on data quality in
    an enterprise setting and what impact it can have. It is valuable to our research 
    because we can relate the findings with what negative impacts poor tickets
    can bring to software projects.
\end{itemize}

\textbf{Metrics for Measuring Data Quality}\cite{Heinrich2007MetricsDataQuality}:
\begin{itemize}
  \item \textbf{What were the goals?}
    The authors of the paper aimed to answer the two main research questions:
      \begin{itemize}
        \item how can data quality be measured as metrics?
        \item what can be done in regards to data quality and what economic
          consequences would it bring?
      \end{itemize}
  \item \textbf{What was the method?}
    They applied the metric for timeliness at a major German mobile services
    provider. Due to some Data Quality issues the company was having, they had
    lower mailing campaign success rates, but after applying the metrics, the 
    company was able to establish a direct connection between the results of
    measuring data quality and the success rates of campaigns.
  \item \textbf{What did they learn?}
    For some practical problems, the metrics that they applied in regards to
    data quality proved to be quite appropriate. As they designed the metrics
    to revolve around interpretability and cardinality, they could quantify
    data quality, thus they could analyse the economical impact.
  \item \textbf{Relevance to our work}: this paper could possibly help us in
    measuring data quality. This is an important step in our research as we 
    specifically need to measure quality in software tickets.
\end{itemize}

\subsection{Issue Quality}

\textbf{Automatic Bug Report Assignment}\cite{anvik2006automating}:
\begin{itemize}
  \item \textbf{What were the goals?}
    The author’s main goal was to determine a way through which bug triaging can be made 
    easier using automatization approaches. 
  \item \textbf{What was the method?}
    The research’s output was an algorithm that can automatically assign bugs to developers based on various types of information fed. The author looked exclusively at ML approaches, examining 8 main types of information (textual description of bug, bug component, OS, hardware, software version, developer who owns the code, current workload of developers, list of devs actively contributing). Moreover, the author considers launching a Firefox extension to test the app even further through human evaluation as well. 
  \item \textbf{What did they learn?}
  The research brings 3 main contributions: methodology for creating semi-automated bug triaging algorithms, a characterization of performance of different approaches, as well as an actual implementation for such a semi-automated big triaget. The author discovered that with further improvements and tweaking, we can reach an even further bug triager.   
  \item \textbf{Relevance to our work}:
  Another project that we can use to learn about bug triaging; also we can contact maybe the author and receive either an executable or the source code for the application. 
\end{itemize}

\textbf{Reducing the effort of bug report triage}\cite{anvik2011reducing}:
\begin{itemize}
  \item \textbf{What were the goals?}
  The authors had one main goal, and that was to find out if costs related to bug triaging can be reduced if ML algorithms were to be introduced. This hypothesis was backed by 2 research questions, which were if the approach would create recommenders that make accurate recommendations and if the humans can make use of the information provided by the approach. 
  \item \textbf{What was the method?}
  They created a bug triager which can present a list of developers from which someone can choose who is the most suitable to fix that specific issue. Also, they improved a previous research the same authors worked on, thus increasing the precision and recall of the algorithm. They employed machine learning techniques to create the algorithm. Then, in order to test it, they implemented a proxy to the actual web service that’s providing the issue repository, thus visualizing results helped conduct the evaluation process.
  \item \textbf{What did they learn?}
  The authors learned that their approach could actually be used in software projects as its accuracy is reasonable. They also learned that the impact of poor software project management, including bug triaging, on software projects can be  quite high.
  \item \textbf{Relevance to our work}:
  The research presented can help us in understanding better if statistical expetiments are helpful in analyzing tickets, as well as give an insight into how important ticket quality and proper triaging are for any software projct. 
\end{itemize}

\textbf{Effects of process maturity on quality, cycle time and effort in software product development}\cite{harter2000effects}:
\begin{itemize}
  \item \textbf{What were the goals?}
  The research tries to investigate the relationship between process maturity, quality, cycle time and effort in software projects. The authors are mainly trying to prove/reject a couple of hypotheses: higher levels of process maturity lead to higher product quality in software projects, higher levels of process maturity are associated with increased cycle time in software products, higher product quality is associated with lower cycle time, higher levels of process maturity lead to increased development efforts in the project and higher product quality is associated with lower development effort.
  \item \textbf{What was the method?}
  In order to test their hypotheses, the authors examined data coming from 30 projects belonging to the systems integration division from a large IT company. The process improvement data was collected via external divisions and by government agencies to provide independent assessments of the firm’s software development processes. Then, they analyzed a couple of key variables and see how they interact: process maturity, product quality, cycle time, development effort, product size, domain/data/decision complexity and requirements ambiguity.
  \item \textbf{What did they learn?}
  They learned that the main features they inspected are additively separable and linear. Moreover, they found out that higher levels of process maturity are associated with significantly higher quality, but also with increased cycle times and development efforts. On the other hand, the reductions in cycle time and effort resulting from improved quality outweigh the marginal increases from achieving more process maturity.
  \item \textbf{Relevance to our work}:
  It can aid our research through better understanding of how various factors can influence the software development processes, thus giving more value to our own research (i.e. being able to automatically determine issue quality could improve better issue writing guidelines/better rules to be enforced, thus reduced triaging and development costs). 
\end{itemize}

\textbf{Where should the bugs be fixed?-more accurate information retrieval-based bug localization based on bug reports}\cite{zhou2012should}:
\begin{itemize}
  \item \textbf{What were the goals?}
  The main goal of the paper was to implement a tool that, based on a bug report, could accurately select a number of files where the developer needs to make the necessary changes to fix the issue. They tried to answer 4 main research questions: how many bugs can be successfully lovated by BugLocator, does the revised VSM improve bug localization accuracy, does considering similar bugs improve localization accuracy, can bugLocator outperform other similar methods. 
  \item \textbf{What was the method?}
  The tool firstly performed a textual analysis, looking for similarities between the bug report description and the source code files. Then, it analyzed previous bugs in the repository to find the most similar ones, thus being able to find which files ought to be changed. Lastly, it assigned scores to similar files, the ones with bigger sizes obtaining higher scores as they are more likely to contain bugs. They collected the necessary data from Bugzilla projects and then performed the evaluation. 
  \item \textbf{What did they learn?}
  The buglocator can locate a large percentage of bugs analysing just a small set of source code files. Secondly, the revised VSM outperforms the standard VSM. Moreover, similar bugs can improve the localization accuracy only to a certain extent, while the locator outperformed every other competitor on a multitude of projects. 
  \item \textbf{Relevance to our work}:
  We can try to use BugLocator and even include it in our research as it gives a direct correlation between bug reports and development effort (thus reduced costs). Thus, we can infer even more possible hypotheses regarding our issues’ quality. 
\end{itemize}

\textbf{}\cite{}:
\begin{itemize}
  \item \textbf{What were the goals?}
  \item \textbf{What was the method?}
  \item \textbf{What did they learn?}
  \item \textbf{Relevance to our work}:
\end{itemize}

\textbf{}\cite{}:
\begin{itemize}
  \item \textbf{What were the goals?}
  \item \textbf{What was the method?}
  \item \textbf{What did they learn?}
  \item \textbf{Relevance to our work}:
\end{itemize}

\textbf{}\cite{}:
\begin{itemize}
  \item \textbf{What were the goals?}
  \item \textbf{What was the method?}
  \item \textbf{What did they learn?}
  \item \textbf{Relevance to our work}:
\end{itemize}

\textbf{}\cite{}:
\begin{itemize}
  \item \textbf{What were the goals?}
  \item \textbf{What was the method?}
  \item \textbf{What did they learn?}
  \item \textbf{Relevance to our work}:
\end{itemize}

\textbf{}\cite{}:
\begin{itemize}
  \item \textbf{What were the goals?}
  \item \textbf{What was the method?}
  \item \textbf{What did they learn?}
  \item \textbf{Relevance to our work}:
\end{itemize}

\textbf{}\cite{}:
\begin{itemize}
  \item \textbf{What were the goals?}
  \item \textbf{What was the method?}
  \item \textbf{What did they learn?}
  \item \textbf{Relevance to our work}:
\end{itemize}


\textbf{Bug Report Assignee Recommendation using 
Activity Profiles}\cite{Naguib2013BugReportAssignee}:
\begin{itemize}
  \item \textbf{What were the goals?}
    The main goal of the paper is to automatically detect who should be
    the developers who are most suitable to solve a specific issue.
  \item \textbf{What was the method?}
    They employed an algorithm using activity profiles (i.e.\ assign, review,
    resolve activity of the developer) such that, after detecting the prior
    experience, developer's role, and involvement in the project, it could
    recommend who should fix a specific bug. The average accuracy was around
    88\%, much higher than the LDA-SVM technique.
  \item \textbf{What did they learn?}
    They found out that their approach was more accurate than previous work
    done on assignee recommendation, but they think that it might even be
    improved further by combining their technique with the classic LDA-SVM one.
  \item \textbf{Relevance to our work}: this work complements the research
    presented in Anvik et al. \cite{anvik2006should} and it will provide us
    with valuable insight into how automatic triaging should be performed.
\end{itemize}

\textbf{Modeling Bug Report Quality}\cite{hooimeijer2007modeling}
\begin{itemize}
  \item \textbf{What were the goals?}
    The authors' goal is to present a model that would determine if a bug
    report will be triaged in a given amount of time. Moreover, they are 
    trying to propose features that could help composing a good quality
    bug report.
  \item \textbf{What was the method?}
    They run an analysis on over 27000 bug reports from the Mozilla Firefox
    project, selecting a couple of surface features that could be applied:
      \begin{itemize}
        \item self-reported severity;
        \item readability;
        \item daily load (i.e.\ total number of bug reports that need to be
          dealt with at that point in time);
        \item submitter reputation;
        \item changes over time.
      \end{itemize}
    Afterwards, they ran 4 experiments (i.e.\ validate the assumptions, find
    how much post-submission data is necessary, find the optimal resolved by
    cutoff, evaluate the hypothetical benefit) and analyzed the results to
    see if their model would be beneficial.
  \item \textbf{What did they learn?}
    They found out that linear model, even a simple one, can have 
    better-than-chance predictive power. Also, they came to the conclusion that 
    attachment and comment counts are really valuable for triaging the bug faster,
    compared to, for example, patch count. Moreover, the readability of a bug is
    important as it proved to influence the duration of the triaging process. 
  \item \textbf{Relevance to our work}: the authors' goal was quite similar to ours
    as they are trying to find out which features make for a good quality bug 
    report. Moreover, they are also using programmatic analysis instead of
    interviews with developers to produce the results, which is the same approach
    we are aiming for in our research.
\end{itemize}

\textbf{Secret Life of Bugs}\cite{aranda2009secret}:
\begin{itemize}
\item What were the goals?
  The paper tries to understand and analyze common bug fixing coordination
  activities. Another goal of the paper was to analyze the reliability of
  repositories in terms of software projects coordination and propose
  different directions on how to implement proper tools.
\item What was the method?
  They executed a field study which was split into two parts:
  \begin{itemize}
  \item firstly, they did an exploratory case study of bug repos histories;
  \item secondly, they conducted a survey with professionals (i.e.\ testers,
    developers).
  \end{itemize}
  All data and interviews were conducted using Microsoft bug repositories and
  employees.
\item What did they learn?
  They learned that there are multiple factors which influence the coordination
  activities that revolve around bug fixing, such as organizational, social and
  technical knowledge, thus one cannot infer any conclusions only by automatic
  analysis of the bug repositories. Also, through surverying the professionals,
  they reached the conclusion that there are 8 main goals which can be used for
  better tools and practices:
  \begin{itemize}
  \item probing for ownership;
  \item summit;
  \item probing for expertise;
  \item code review;
  \item triaging;
  \item rapid-fire emailing;
  \item infrequent, direct email;
  \item shotgun emails.
  \end{itemize}
\item Relevance to our work: this paper is one of the key papers for our
  research paper.
\end{itemize}

\textbf{What makes a good bug report}\cite{bettenburg2008makes}:
\begin{itemize}
  \item \textbf{What were the goals?}
  The main goal of the paper was to investigate the \textbf{quality of bug reports} from a developer's
  point of view, based on the typical information found in such a report (e.g.\ stack traces, screenshots).
\item \textbf{What was the method?}
  The authors conducted a massive survey with over 450 respondents. The survey was online and it targeted
  developers from Mozilla, Apache and Eclipse.
\item \textbf{What did they learn?}
  The main conclusion of this research paper was that well written bug reports will more likely get the 
  attention of the developers. Thus, including steps to reproduce the bugs or stack traces proved to increase
  the quality of the bug report. Also, an important achievement reached by the authors was the development of
  a prototype tool called Cuezilla that could estimate, with an accuracy rate of 31\-48\%, the quality of a bug
  report.
\item \textbf{Relevance to our work}: it is relevant to our research in that it provides valuable insight into 
  what makes for a good bug report based on actual professionals' opinions.
\end{itemize}

\textbf{Who should fix this bug?}\cite{anvik2006should}:
\begin{itemize}
  \item \textbf{What were the goals?}
    The authors aimed to create a tool that could automatically assign the 
    bug report to a specific developer based on his/her suitability for that
    specific task.
  \item \textbf{What was the method?}
    They applied a supervised machine learning algorithm on the 
    repositories to learn which developers were best suited for specific tasks,
    thus when a new bug report would come in, a small set of people 
    would be selected. In order to train the algorithm, they looked at
    Bugzilla repositories and selected the free text form of tickets, trying
    to label similar ones based on textual similarities. Once the tickets were 
    labeled and grouped for specific developers, the algorithm would then be
    able to present the triager the set of developers suitable to fix the bug.
  \item \textbf{What did they learn?}
    The most important lesson learned was that collecting data from bug reports 
    and CVS logs was quite challenging. One of the major reasons why they 
    found this aspect hard was that not all CVS comments referenced the 
    specific bug report id.
  \item \textbf{Relevance to our work}
    The paper taught us that bug triaging is hard and that there is almost no 
    automated tool that can choose the perfect developers to work on the task.
    Moreover, the method applied by the authors could prove useful as a 
    learning aid when working with the open source repositories 
    chosen as data sets.
\end{itemize}

\textbf{Software Quality \- The Elusive Target}\cite{kitchenham1996software}:
\begin{itemize}
  \item \textbf{What were the goals?}
    The main goal of the paper is to determine what makes for a good quality
    software project, as well as who are the people in charge of this aspect
    and how should they approach achieving it.
  \item \textbf{What was the method?}
    They tried to define quality in software projects and analyze techniques that 
    measure such quality by looking at other models proposed in different other  
    papers (e.g.\ McCall's quality model, ISO 9126).
  \item \textbf{What did they learn?}
    They learned that quality is very hard to define and there are various 
    factors which need to be taken into consideration, such as the business
    model of the company, the type of the software project (e.g.\ safety
    critical, financial) or the actors which are involved and how they
    coordinate the software activities.
  \item \textbf{Relevance to our work}: this paper is a key paper on software  
    quality and it can prove beneficial in our research by
    giving valuable insights into how software quality can be modeled, thus
    helping us in selecting good quality open source repositories to work with 
    (i.e.\ ticket selection and analysis).
\end{itemize}

\textbf{Code Quality Analysis in OSS}\cite{stamelos2002code}:
\begin{itemize}
  \item \textbf{What were the goals?}
    The article tries to discuss and examine the quality of the source code
    delivered by open source projects.
  \item \textbf{What was the method?}
    They used a set of tools that could automatically inspect various aspects
    of source code. The authors analyzed the 6th release of the OpenSUSE project
    and examined only the components, which are defined by C functions in the
    programs.
  \item \textbf{What did they learn?}
    The research's results show that Linux applications have high quality 
    code standards that one might expect in an open source repository, but 
    the quality is lower than the one implied by the standard. More than half
    of the components were in a high state of quality, but on the other hand, 
    most lower quality components cannot be improved only by applying some
    corrective actions. Thus, even though not all the source code was in 
    an industrial standards shape, there is definitely room for further 
    improvement and open source repositories proved to be of good quality.
  \item \textbf{Relevance to our work}: this work completes the previous paper
    on software quality in general by looking specifically at quality in
    open source projects, which will be our main points for data collection.
\end{itemize}

\textbf{Analysis of Software Cost Estimation}\cite{grimstad2006framework}:
\begin{itemize}
  \item \textbf{What were the goals?}
    The authors are trying to show that poor estimation analysis techniques 
    in software projects will lead to wrong conclusions regarding cost 
    estimation accuracy. Moreover, they also propose a framework for 
    better analysis of software cost estimation error.
  \item \textbf{What was the method?}
    They approached a real-world company where they conducted analysis on
    their cost estimation techniques.
  \item \textbf{What did they learn?}
    They learned that regular, straight-forward types of cost estimation analysis
    techniques error lead them to wrong conclusions. 
  \item \textbf{Relevance to our work}: it showed us that we need to be careful
    when selecting techniques for cost estimation in our own research.
\end{itemize}

\subsection{Measuring Cost and Waste in Software Projects}

\textbf{Waste Identification}\cite{Korkala2014WasteIdentification}:
\begin{itemize}
  \item \textbf{What were the goals?}
    The paper had two main goals: 
    \begin{itemize}
      \item a means to identify communication waste in agile software projects 
        environments;
      \item types of communication waste in agile projects.
    \end{itemize}
  \item \textbf{What was the method?}
    The authors collaborated with a medium-sized American software company
    and conducted a series of observations, informal discussions, documents 
    provided by the organization, as well as semi-structured interviews. Moreover,
    the data collection for waste identification was split into 2 parts:
      \begin{itemize}
        \item \textbf{pre-development}: occured before the actual implementation
          begun (e.g.\ backlog creation);
        \item \textbf{development}: happened throughout the implementation
          process (e.g.\ throughout sprints, retrospectives, sprint reviews,
          communication media).
      \end{itemize}
  \item \textbf{What did they learn?}
    They realized the communication waste can be divided into 5 main categories:
      \begin{itemize}
        \item lack of involvement;
        \item lack of shared understanding;
        \item outdated information;
        \item restricted access to information;
        \item scattered information.
      \end{itemize}
    Also, they learned that their way of identifying these types of waste was
    quite efficient and they even recommend it to companies if they'd like
    to conduct such processes internally.
  \item \textbf{Relevance to our work}: the waste identification process can
    be applied to our work so that we can identify possible causes to poor 
    quality tickets.
\end{itemize}

\textbf{Software Development Waste}\cite{sedano2017software}:
\begin{itemize}
  \item \textbf{What were the goals?}
    The main goal of the paper was to identify main types of waste in software
    development projects.
  \item \textbf{What was the method?}
    They conducted a participant-observation study over a long period of time
    at Pivotal, a consultancy software development company. They also interviewed
    multiple engineers and balanced theoretical sampling with analysis to achieve
    the conclusions.
  \item \textbf{What did they learn?}
    They found out there are nine main types of waste in software projects:
    \begin{itemize}
      \item building the wrong feature or product;
      \item mismanaging backlog;
      \item extraneous cognitive load;
      \item rework;
      \item ineffective communication;
      \item waiting/multitasking;
      \item solutions too complex;
      \item psychological distress.
      \item \textbf{Relevance to our work}: this paper complements the previous 
    one on waste identification\cite{Korkala2014WasteIdentification}.
    \end{itemize}
\end{itemize}

\textbf{Waste in Kanban Projects}\cite{ikonen2010exploring}:
\begin{itemize}
  \item \textbf{What were the goals?}
    The authors are trying to find the main sources of waste in Kanban software
    development projects and categorize/rank them based on severity.
  \item \textbf{What was the method?}
    A controlled case study research was employed in a company called Software
    Factory. They conducted semi-structured interviews with 5 of the team
    members both in the beginning in order to collect data as well as at the end 
    of the whole process to categorize the seven types of waste found. Moreover,
    they also measured the overall success of the project based on Shenar's
    techniques (first-second-third-fourth; project efficiency-imapct on the
    customer-business success-preparing for the future).
  \item \textbf{What did they learn?}
    They reached two main findings:
      \begin{itemize}
        \item they found 7 types of waste throughout the project at various
          development stages:
            \begin{itemize}
              \item partially done work;
              \item extra processes;
              \item extra features;
              \item task switching;
              \item waiting;
              \item motion;
              \item defects.
            \end{itemize}
        \item they reached the conclusion that they couldn't explain the success
          of the project even though waste was found.
      \end{itemize}
  \item \textbf{Relevance to our work}: this work completes the findings 
    from the previous work presented as most of the projects we will work with
    will be Agile, thus Kanban-based in terms of issue management.
\end{itemize}


\subsection{Sentiment Analysis}


\textbf{Thumbs Up or Thumbs Down}\cite{turney2002thumbs}:
\begin{itemize}
  \item \textbf{What were the goals?}
    The main goal of the paper is to detect the overall sentiment transmitted
    through reviews of various types.
  \item \textbf{What was the method?}
    The author created an unsupervised machine learning algorithm that was 
    evaluated on more than 400 reviews on Epinions on various kinds of markets 
    (e.g.\ automobiles, movie). The algorithm implementation was divided 
    into three steps:
      \begin{itemize}
        \item extract phrases containing adjectives or adverbs;
        \item estimate the semantic orientation of the phrases;
        \item classify the review as recommended or not recommended based on
          the semantic orientation calculated at previous step.
      \end{itemize}
  \item \textbf{What did they learn?}
    One thing the author learned that different categories will yield different 
    results. For example, the automobile section on Epinions ranked much higher, 
    84\%, compared to movie reviews, which had an accuracy of 65.83\%. Moreover, 
    most pitfalls of the algorithm could be attributed to multiple factors, such 
    as not using a supervised learning system or limitations of PMI-IR.
  \item \textbf{Relevance to our work}: the method can be applied for extracting
    sentiments from the tickets (description and comments) we will use 
    in our own research.
\end{itemize}

\textbf{Recognizing Contextual Polarity}\cite{wilson2005recognizing}:
\begin{itemize}
  \item \textbf{What were the goals?}
    The paper's main goal is to find efficient ways to distinguish between 
    contextual and prior polarity.
  \item \textbf{What was the method?}
    They used a two step method that used machine learning and a variety of
    features. The first step classified each phrase which had a clue as either
    neutral or polar, followed by taking all phrases marked in the previous step
    and giving them a contextual polarity (e.g.\ positive, negative, 
    both, neutral).
  \item \textbf{What did they learn?}
    Through the method the authors employed, they managed to automatically 
    identify the contextual polarity. As most papers were only looking at the 
    sentiment extracted from the overall document, they managed to get valuable 
    results from looking at specific words and phrases.
  \item \textbf{Relevance to our work}: when analyzing the description and
    comments of the ticket, we can use their method for infering the sentiment 
    transmitted probably more accurately than the technique used in 
    Turney's paper\cite{turney2002thumbs}.
\end{itemize}

\subsection{Conclusion}

%%%%%%%%%%%%%%%%%%%%%%%%%%%%%%%%%%%%%%%%%%%%%%%%%%%%%%%%%%%%%%%%%%%
\section{Proposed Approach}

state how you propose to solve the software development problem. Show that your proposed approach is feasible, but identify any risks.

%%%%%%%%%%%%%%%%%%%%%%%%%%%%%%%%%%%%%%%%%%%%%%%%%%%%%%%%%%%%%%%%%%%
\section{Work Plan}

show how you plan to organize your work, identifying intermediate deliverables and dates.

%%%%%%%%%%%%%%%%%%%%%%%%%%%%%%%%%%%%%%%%%%%%%%%%%%%%%%%%%%%%%%%%%%%
\bibliographystyle{plain}
\bibliography{mprop}
\end{document}
