\documentclass{mprop}
\usepackage{graphicx}

% alternative font if you prefer
%\usepackage{times}

% for alternative page numbering use the following package
% and see documentation for commands
%\usepackage{fancyheadings}


% other potentially useful packages
%\uspackage{amssymb,amsmath}
%\usepackage{url}
%\usepackage{fancyvrb}
%\usepackage[final]{pdfpages}

\begin{document}

%%%%%%%%%%%%%%%%%%%%%%%%%%%%%%%%%%%%%%%%%%%%%%%%%%%%%%%%%%%%%%%%%%%
\title{Software Ticket Quality}
\author{Andrei-Mihai Nicolae}
\date{December 18th 2017}
\maketitle
%%%%%%%%%%%%%%%%%%%%%%%%%%%%%%%%%%%%%%%%%%%%%%%%%%%%%%%%%%%%%%%%%%%

%%%%%%%%%%%%%%%%%%%%%%%%%%%%%%%%%%%%%%%%%%%%%%%%%%%%%%%%%%%%%%%%%%%
\tableofcontents
\newpage
%%%%%%%%%%%%%%%%%%%%%%%%%%%%%%%%%%%%%%%%%%%%%%%%%%%%%%%%%%%%%%%%%%%

%%%%%%%%%%%%%%%%%%%%%%%%%%%%%%%%%%%%%%%%%%%%%%%%%%%%%%%%%%%%%%%%%%%
\section{Introduction}\label{intro}

briefly explain the context of the project problem

\subsection{A subsection}
Please note your proposal need not follow the included section headings; this is only a suggested structure. Also add subsections etc as required

%%%%%%%%%%%%%%%%%%%%%%%%%%%%%%%%%%%%%%%%%%%%%%%%%%%%%%%%%%%%%%%%%%%
\section{Statement of Problem}

Clearly state the problem to be addressed in your forthcoming project.
Explain why it would be worthwhile to solve this problem.

%%%%%%%%%%%%%%%%%%%%%%%%%%%%%%%%%%%%%%%%%%%%%%%%%%%%%%%%%%%%%%%%%%%
\section{Background Survey}

\subsection{Introduction}

\subsection{What was covered and why}

\subsection{How papers were found; what terms were searched for + snowballing}

\subsection{Data Quality Metrics}

\subsection{Issue Quality}

Secret Life of Bugs\cite{aranda2009secret}:
\begin{itemize}
\item What were the goals?
  The paper tries to understand and analyze common bug fixing coordination
  activities. Another goal of the paper was to analyze the reliability of
  repositories in terms of software projects coordination and propose
  different directions on how to implement proper tools.
\item What was the method?
  They executed a field study which was split into two parts:
  \begin{itemize}
  \item firstly, they did an exploratory case study of bug repos histories;
  \item secondly, they conducted a survey with professionals (i.e.\ testers,
    developers).
  \end{itemize}
  All data and interviews were conducted using Microsoft bug repositories and
  employees.
\item What did they learn?
  They learned that there are multiple factors which influence the coordination
  activities that revolve around bug fixing, such as organizational, social and
  technical knowledge, thus one cannot infer any conclusions only by automatic
  analysis of the bug repositories. Also, through surverying the professionals,
  they reached the conclusion that there are 8 main goals which can be used for
  better tools and practices:
  \begin{itemize}
  \item probing for ownership;
  \item summit;
  \item probing for expertise;
  \item code review;
  \item triaging;
  \item rapid-fire emailing;
  \item infrequent, direct email;
  \item shotgun emails.
  \end{itemize}
\item Relevance to our work: this paper is one of the key papers for our
  research paper.
\end{itemize}

What makes a good bug report\cite{bettenburg2008makes}:
\begin{itemize}
  \item \textbf{What were the goals?}
  The main goal of the paper was to investigate the \textbf{quality of bug reports} from a developer's
  point of view, based on the typical information found in such a report (e.g. stack traces, screenshots).
\item \textbf{What was the method?}
  The authors conducted a massive survey with over 450 respondents. The survey was online and it targeted
  developers from Mozilla, Apache and Eclipse.
\item \textbf{What did they learn?}
  The main conclusion of this research paper was that well written bug reports will more likely get the 
  attention of the developers. Thus, including steps to reproduce the bugs or stack traces proved to increase
  the quality of the bug report. Also, an important achievement reached by the authors was the development of
  a prototype tool called Cuezilla that could estimate, with an accuracy rate of 31\-48\%, the quality of a bug
  report.
\item \textbf{Relevance to our work}: it is relevant to our research in that it provides valuable insight into 
  what makes for a good bug report based on actual professionals' opinions.
\end{itemize}

Who should fix this bug\cite{anvik2006should}:
\begin{itemize}
  \item \textbf{What were the goals?}
    The authors aimed to create a tool that could automatically assign the 
    bug report to a specific developer based on his/her suitability for that
    specific task.
  \item \textbf{What was the method?}
    They applied a supervised machine learning algorithm on the 
    repositories to learn which developers were best suited for specific tasks,
    thus when a new bug report would come in, a small set of people 
    would be selected. In order to train the algorithm, they looked at
    Bugzilla repositories and selected the free text form of tickets, trying
    to label similar ones based on textual similarities. Once the tickets were 
    labeled and grouped for specific developers, the algorithm would then be
    able to present the triager the set of developers suitable to fix the bug.
  \item \textbf{What did they learn?}
    The most important lesson learned was that collecting data from bug reports 
    and CVS logs was quite challenging. One of the major reasons why they 
    found this aspect hard was that not all CVS comments referenced the 
    specific bug report id.
  \item \textbf{Relevance to our work}
    The paper taught us that bug triaging is hard and that there is almost no 
    automated tool that can choose the perfect developers to work on the task.
    Moreover, the method applied by the authors could prove useful as a 
    learning aid when working with the open source repositories 
    chosen as data sets.
\end{itemize}

\citep{aranda2009secret}
\begin{itemize}
  \item \textbf{What were the goals?}
  \item \textbf{What was the method?}
  \item \textbf{What did they learn?}
  \item \textbf{Link to other work}
  \item \textbf{Relevance to our work}
\end{itemize}

\subsection{Measuring Cost and Waste in Software Projects}

Waste Identification\cite{Korkala2014WasteIdentification}:
\begin{itemize}
  \item \textbf{What were the goals?}
    The paper had two main goals: 
    \begin{itemize}
      \item a means to identify communication waste in agile software projects 
        environments;
      \item types of communication waste in agile projects.
    \end{itemize}
  \item \textbf{What was the method?}
    The authors collaborated with a medium-sized American software company
    and conducted a series of observations, informal discussions, documents 
    provided by the organization, as well as semi-structured interviews. Moreover,
    the data collection for waste identification was split into 2 parts:
      \begin{itemize}
        \item \textbf{pre-development}: occured before the actual implementation
          begun (e.g.\ backlog creation);
        \item \textbf{development}: happened throughout the implementation
          process (e.g.\ throughout sprints, retrospectives, sprint reviews,
          communication media).
      \end{itemize}
  \item \textbf{What did they learn?}
    They realized the communication waste can be divided into 5 main categories:
      \begin{itemize}
        \item lack of involvement;
        \item lack of shared understanding;
        \item outdated information;
        \item restricted access to information;
        \item scattered information.
      \end{itemize}
    Also, they learned that their way of identifying these types of waste was
    quite efficient and they even recommend it to companies if they'd like
    to conduct such processes internally.
  \item \textbf{Relevance to our work}: the waste identification process can
    be applied to our work so that we can identify possible causes to poor 
    quality tickets.
\end{itemize}

\subsection{Sentiment Analysis}



\subsection{Conclusion}

%%%%%%%%%%%%%%%%%%%%%%%%%%%%%%%%%%%%%%%%%%%%%%%%%%%%%%%%%%%%%%%%%%%
\section{Proposed Approach}

state how you propose to solve the software development problem. Show that your proposed approach is feasible, but identify any risks.

%%%%%%%%%%%%%%%%%%%%%%%%%%%%%%%%%%%%%%%%%%%%%%%%%%%%%%%%%%%%%%%%%%%
\section{Work Plan}

show how you plan to organize your work, identifying intermediate deliverables and dates.

%%%%%%%%%%%%%%%%%%%%%%%%%%%%%%%%%%%%%%%%%%%%%%%%%%%%%%%%%%%%%%%%%%%
\bibliographystyle{plain}
\bibliography{mprop}
\end{document}
