\documentclass{mprop}
\usepackage{graphicx}

\usepackage{natbib}

% alternative font if you prefer
%\usepackage{times}

% for alternative page numbering use the following package
% and see documentation for commands
%\usepackage{fancyheadings}


% other potentially useful packages
%\uspackage{amssymb,amsmath}
%\usepackage{url}
%\usepackage{fancyvrb}
%\usepackage[final]{pdfpages}

\begin{document}

%%%%%%%%%%%%%%%%%%%%%%%%%%%%%%%%%%%%%%%%%%%%%%%%%%%%%%%%%%%%%%%%%%%
\title{Software Ticket Quality}
\author{Andrei-Mihai Nicolae}
\date{December 18th 2017}
\maketitle
%%%%%%%%%%%%%%%%%%%%%%%%%%%%%%%%%%%%%%%%%%%%%%%%%%%%%%%%%%%%%%%%%%%

%%%%%%%%%%%%%%%%%%%%%%%%%%%%%%%%%%%%%%%%%%%%%%%%%%%%%%%%%%%%%%%%%%%
\tableofcontents
\newpage
%%%%%%%%%%%%%%%%%%%%%%%%%%%%%%%%%%%%%%%%%%%%%%%%%%%%%%%%%%%%%%%%%%%

%%%%%%%%%%%%%%%%%%%%%%%%%%%%%%%%%%%%%%%%%%%%%%%%%%%%%%%%%%%%%%%%%%%

\section{Introduction}\label{intro}

% 2 - 3 pages tops

\begin{itemize}
\item Background briefly explain the context of the project problem
\item Problem statement and justification. Clearly state the problem to be addressed in your forthcoming project.
  Explain why it would be worthwhile to solve this problem.

\item Research questions and contributions
\item Structure of the rest of the document
\end{itemize}

%%%%%%%%%%%%%%%%%%%%%%%%%%%%%%%%%%%%%%%%%%%%%%%%%%%%%%%%%%%%%%%%%%%


%%%%%%%%%%%%%%%%%%%%%%%%%%%%%%%%%%%%%%%%%%%%%%%%%%%%%%%%%%%%%%%%%%%
\section{Background Survey}
% 10 - 12 pages

\subsection{Study Design}

In order to conduct a thorough background survey, several topics needed to be 
explored, among which the most important were:
  \begin{itemize}
    \item data quality and metrics;
    \item issue/ticket quality;
    \item waste in software projects;
    \item sentiment analysis.
  \end{itemize}

Firstly, we will deal with a large volume of tickets and information extracted 
from software projects and the tools that revolve around them, such as issue 
trackers and version control systems. Thus, \emph{data quality} is an important 
topic that needed to be explored in order to infer appropriate conclusions. 

We analyzed data quality mainly in software organizations and looked at what metrics
are most useful when trying to perform measurements. We also tried to find different
methodologies that either measure data quality or calculate the impact of poor data
quality on the software project. 

Secondly, probably the topic of most interest to our research is \emph{issue quality}.
This field is mainly concerned with software tickets and their characteristics: how
important are issue trackers to a software project, what makes for a good bug report,
what impact does bad triaging have etc. As we are interested in finding out what
features are most important in improving the quality of a ticket, the issue quality
literature I have reviewed proved to be beneficial and provided valuable insight into
how other researchers tried to approach this problem.

The third topic was waste in software projects and how it impacts them. We needed to
look into various communication and tool-related waste in software projects because
having a project with a considerable amount of waste might affect our results and
compromise the evaluation process. We investigated researches that looked into what
are the main types of waste in software projects and how it can be avoided so that
we could apply the findings into selecting the projects.

Last but not least, we did a literature review on sentiment analysis as well 
because we will inspect, among other components of a ticket, comments - we want 
to analyze the textual representation of comments and see if the overall 
sentiment influences the time it takes to close a ticket (i.e. maybe a negative 
conversation in comments might increase the time it takes to solve the issue).

The main sources of information when searching for papers were Google Scholar, 
ACM digital library and IEEE Xplore digital library. The first step in the 
literature review process was finding relevant papers that could aid us in
our research and splitting them into the categories mentioned above. There were
a couple of key search phrases that I started with (along with slight variations):
  \begin{itemize}
    \item software data quality;
    \item software data quality metrics;
    \item data quality in software projects;
    \item issue quality in software projects;
    \item ticket quality software;
    \item sentiment analysis;
    \item software project communication waste.
  \end{itemize}

Then, after reading the papers retrieved when searching for the above-mentioned
phrases, I also applied the snowballing research technique as per Dr. Storer's 
advice. This extra step was helpful and allowed me to easily find linked 
papers and correlate them in this literature review.

Moreover, when reading papers, one key aspect for selecting and including it
in the literature review was citation count. Thus, after finding the most 
cited papers, I then checked to see the journal where it was published so that
we would only have material from the most important journals.

Even though the literature review contains around 40 papers, there were
around 80 papers that were reviewed but roughly half of them were excluded 
because they eventually proved to not be so relevant to our research. 

\subsection{Data Quality and Metrics}

Several authors have investigated the \emph{meaning} of data quality and
what characteristics define it.

\citet{bachmann2009software} conducted a thorough investigation of several
software projects, both open source (5 projects) as well as closed source (1 
project), in order to infer what determines if it has quality. They selected 
various sources of information, among which bug tracking 
databases and version control systems logs, and examined SVN logs, CVS logs
and the content of the bug tracker databases, in the end trying to link 
the logs with the bug tracker contents as they are not integrated by default. 
On the other hand, \citet{strong1997data} conducted a qualitative analysis
instead of quantitative by collecting data from 42 data quality projects by 
interviewing custodians, customers and managers. They analyzed each project
using the data quality dimensions as content analysis codes.

The first study came to several conclusions, among which:
  \begin{itemize}
    \item closed source software projects usually exhibit better data quality 
    (i.e. better average number of attachments, better average status changes, 
    better average number of commits per bug report);
    \item reduced traceability between bug reports and version control logs
    due to scarce linkage between the two;
    \item open source projects exhibit reduced quality in change logs as,
    for example, the Eclipse project has over 20\% empty commit messages.
  \end{itemize}

However, the second study reached the conclusion that representational data 
quality dimensions are underlying causes of accessibility data quality problem
patterns. The authors also found out that three underlying causes for users'
complaints regarding data not supporting their tasks are incomplete data, 
inadequately defined or measured data and data that could not be appropriately
aggregated.

After defining what data quality is and what are its characteristics, the next 
step would be how to measure data or information quality in a project. 

There are several research papers that try to find the answer, among which the 
work of \citet{lee2002aimq}. The authors tried to define an overall model along 
with an accompanying assessment instrument for quantifying information quality 
in an organization. The methodology has 3 main steps that need to be followed 
in order to apply it successfully:
  \begin{itemize}
    \item 2 $\times$ 2 model of what information quality means to managers;
    \item questionnaire for measuring information quality along the dimensions
    found in first step;
    \item two analysis techniques for interpreting the assessments captured 
    by the questionnaire.
  \end{itemize} 
After developing the technique, they applied it at 5 different organizations and
found that the tool proved to be practical.

Compared to the methodology proposed by \citet{lee2002aimq}, the solution found
by \citet{Heinrich2007MetricsDataQuality} is quite different. Even though the 
main goals were the same, the authors used a single metric, and that is 
the metric for timeliness (i.e. whether the values of attributes still 
correspond to the current state of their real world counterparts and whether 
they are out of date). Thus, they applied the metric at a major German mobile 
services provider. Due to some Data Quality issues the company was having, they 
had lower mailing campaign success rates, but after applying the metrics, the 
company was able to establish a direct connection between the results of
measuring data quality and the success rates of campaigns.

\citet{nelson2005antecedents} proposed another technique for measuring data 
quality in an organization by, firstly, setting 2 main research questions:
  \begin{itemize}
    \item identify a set of antecedents that both drive information and
      system quality, as well as define the nature of the IT artifact;
    \item explore data warehousing in general, especially analytical tools,
      predefined reports and ad hoc queries.
  \end{itemize}

Then, the authors set up a model to define a tree-structured representation
of system quality and data quality, as follows:
  \begin{itemize}
    \item data quality - defined by completeness, accuracy, format and currency;
    \item system quality - defined by reliability, flexibility, integration,
    response time and accessibility;
    \item then, data and systems available are evaluated to infer 
    data satisfaction and system satisfaction (coming from customers), which
    in turn will compute the final satisfaction score of the product.
  \end{itemize}

After conducting a cross-functional survey to test the model, they learned that
the features they selected were a good indicator of overall information/data and
system quality. They also found that accuracy is the dominant determinant across
all three data warehouse technologies, followed by completeness and format. Last
but not least, another discovery was that more attention needs to be given to
differences across varying technologies.

Another important area of data quality is what methodologies or techniques can 
be applied by organizations in order to improve data quality. 
There are several studies that tried to propose such methodologies. A general 
overview of such techniques  was presented by \citet{pipino2002data} - they 
wanted to present ways of developing usable  data quality methods that 
organizations can implement in their own internal processes. After reviewing 
various techniques of assessing data quality in information systems, they reached 
the conclusion that there is no universal approach to assess data quality as it  
heavily depends on the context where it is analysed. 

One such methodology examined by \citet{pipino2002data} is the technique 
proposed by \citet{wang1998product} called Total Data Quality Management
(abbreviated TDQM). Through this method, the authors wanted to help 
organizations deliver high quality information products to information
consumers by following the TDQM cycle - define, measure, analyse and improve
information quality continuously. In order to do that, the research proposes
4 steps: clearly articulate the information product, establish several roles
that would be in charge of the information product management, teach information
quality assessment and management skills to all information products 
constituencies and, finally, institutionalize continuous information product 
improvement. After developing the methodology, the authors conclude on a 
confident note that their technique can fill a gap in the information quality
environment.

However, there are other aspects that can be investigated when trying to improve
data quality. For example, \citet{prybutok2008evaluating} tried to find if 
leadership and information quality can lead to positive outcomes in an 
e-government's data quality. They conducted a web survey to gather data and test 
their hypotheses. Afterwards, they assessed the City of Denton's e-government 
initiatives, including current plans and implementations. They learned that the 
MBNQA leadership triad (leadership, strategic planning and customer/market focus) 
had a positive impact on the IT quality triad (information, system and service 
quality). Moreover, they found out that both leadership and IT quality improved 
the benefits.

Finally, after defining, measuring and improving data quality, we also need
to be able to assess the impact poor DQ has on a company. In this area, there
are several authors that investigated various implications of low indices
of data quality.

One such research is the one done by \citet{gorla2010organizational}. They wanted
to examine what influences do system quality, service quality and data quality
have on an organization, as well as what effect does system quality have on
data quality. After conducting a construct measurement for the information
systems quality dimensions \citep{swanson1997maintaining} and collecting data
through empirical testing, they learned that service quality has the greatest
impact of all three quality constructs. Moreover, they found out that there is 
a linkage between system and information quality, fact that previous research
did not reveal. Last but not least, their results indicate that the 
above-mentioned quality dimensions have a significant positive influence on 
organizational impact either directly or indirectly. 

On the other hand, \citet{redman1998impact} had different findings. Even though
they did not analyze multiple types of quality, but only data quality, they 
learned that there are three main issues across most enterprises: inaccurate
data, inconsistencies across databases and unavailable data necessary for 
certain operations. Furthermore, they presented a number of impacts that
poor data quality has on enterprises: lowered customer satisfaction, increased
cost, lowered employee satisfaction, it becomes more difficult to set strategy,
as well as worse decision making.

Another research, the one of \citet{lee2003knowing}, investigated another 
aspect: whether knowing-why affects work performance and whether knowledge held
by people in different roles affect the overall work performance. After
conducting various surveys with 6 companies that served as data collection 
points, they reached the conclusion that there is vast complexity of 
knowledge at work in the data production process and if these gaps are not 
bridged it might lead to decrased data quality.

\subsection{Issue Quality}

Issue quality is the main topic of this literature review as it revolves around
software tickets and what are the characteristics of a well-written and 
informative bug report, how efficient are bug tracking systems, what defines
an efficient bug triaging process etc. 

The first and most important sub-topic of issue quality, which is the one
we will address in our own research as well, is the quality of bug reports.
There are several authors that have tried to defined what makes for a good bug
report and what components actually improve the overall quality.

\citet{bettenburg2008makes} have tried, through their work, to analyze what 
makes for a good bug report through qualitative analysis. They interviewed over
450 developers and asked them what are the most important features for them in
a bug report that help them solve the issue quicker. They reached the conclusion 
that stack traces and steps to reproduce increased the quality of a bug report 
the most, followed by well-written, grammar error free summaries and 
descriptions. Last but not least, they also created a tool called CueZilla that 
could with an accuracy rate between 31 and 48\% predict the quality of a bug 
report. 

Strengthening the argument that readability matters considerably in bug report 
quality is the work of \citet{hooimeijer2007modeling}. They ran an analysis over
27000 bug reports from the Mozilla Firefox project, looking at self-reported 
severity, readability, daily load, submitter reputation and changes over time. 
After running the evaluation, they not only found out about the importance of
readability in bug reports, but also that attachment and comment counts are 
valuable for faster triaging and that patch count, for example, does not
contribute in the same manner as the previous two.

Another research that agrees that stack traces are helpful in solving bug
reports faster is the one performed by \citet{schroter2010stack}. They conducted
the whole experiment on the Eclipse project. They first extracted the stack
traces using the infoZilla tool(\citep{bettenburg2008extracting}), followed by
linking the stack traces to changes in the source code (change logs) by mining
the version repository of the Eclipse project. The results were that 
around 60\% of the bugs that contained stack traces in their reports were fixed
in one of the methods in the frame, with 40\% being fixed in exactly the first
stack frame.

Following on the conclusions of \citet{bettenburg2008makes} we have found out
that the investigations undertaken by \citet{bettenburg2007quality} reveal 
very similar results. They performed the same type of evaluation method (i.e. 
interviews with developers) and confirmed that steps to reproduce and stack traces
are the most important features in a bug report that help developers solve the
issue quicker.

However, in order to model the quality of a bug report, one needs to be able to
successfully extract various types of information from such a report and, if 
possible, link them to the source code of the project (i.e. source code fragments
in bug report discussions should be linked to the corresponding sections in the 
actual code) or other software artifacts. There are several researches that tried 
to analyze such techniques and one of them is the work of 
\citet{bettenburg2012using}. The authors created a tool that could parse a bug 
report (using fuzzy code search) and extract source code that could then be
matched with exact locations in the source code, in the end producing a 
traceability link (i.e. tuple containing a clone group ID and file paths 
corresponding to the source code files). Evaluation showed an increase of
roughly 20\% in total traceability links between issue reports and source code 
when compared to the current state-of-the-art technique, change log analysis.

\citet{bettenburg2012using} also made use of a tool developed by 
\citet{bettenburg2008extracting} called infoZilla. This application can
parse bug reports and correctly extract patches, stack traces, source code and
enumerations. When evaluating it on over 160.000 Eclipse bug reports, it proved to 
have a very high rate of over 97\% accuracy.

However, \citet{kim2013should} propose a rather different technique than the one
exposed in \citet{bettenburg2012using}. The authors have employed a two-phase
recommendation model that would locate the necessary fixes based on information
in bug reports. Firstly, they did a feature extraction on the bug reports (e.g. 
extract description, summary, metadata). Then, in order to successfully predict
locations, this model was trained on collected bug reports and then, when given
a new bug report, it would try to localize the files that need to be changed 
automatically. Finally, the actual implementation was put into place, and that
is the two phase recommendation model, composed of binary (filters out 
uninformative bug reports before predicting the files to fix) and multiclass (
previously filtered bug reports are used as training data and only after that
new bug reports can be analyzed and files to be changed recommended). The 
overall accuracy was above the one achieved by \citet{bettenburg2012using}, 
being able to rank over 70\%, but only as long as it recommended \emph{any} 
locations.

Even though we have discussed about quality in bug reports and how it can be
identified, we must also investigate if the platform where these bug reports 
reside (i.e. bug trackers) are properly fitting the software environment. 
The work of \citet{just2008towards} tries to find the answer to this question.
The authors launched a survey to developers from Eclipse, Mozilla and Apache
open source projects from which they received 175 comments back. Then, they
applied a card sort in order to organize the comments into hierarchies to 
deduce a higher level of abstraction and identify common patterns. In the end,
they ranked the top suggestions as follows:
  \begin{itemize}
    \item provide tool support for users to collect and prepare information that 
      developers need;
    \item find volunteers to translate bug reports filed in foreign languages;
    \item provide different user interfaces for each user level and give cues to 
      inexperienced reporters on what to report and best practices;
    \item reward reporters when they do a good job;
    \item integrate reputation into user profiles to mark experienced reporters;
    \item provide powerful and easy to use tools to search bug reports;
    \item encourage users to submit additional details (provide tools for merging 
      bugs).
  \end{itemize}

One other important aspect when analyzing the quality of a bug report is the
time it takes to triage it. Triaging is the process of analyzing issues and 
based on the information available, assign it to 

\textbf{Predicting the severity of a reported bug}\cite{lamkanfi2010predicting}:
\begin{itemize}
  \item \textbf{What were the goals?}
    The main goal of the research was to determine if, using textual analysis, one can
    predict the severity of a bug. This main research question was further divided into
    4 RQs:
      \begin{itemize}
        \item which terms in the textual description could serve as good indicators for
        the bug severity;
        \item what types of textual fields (e.g. summary, description) serve indicate the
        bug severity best;
        \item how many sample does one need to train an accurate predictor;
        \item is it better to have a specialized predictor for each component or combine
        bug reports over multiple components.
      \end{itemize}
  \item \textbf{What was the method?}
    The authors conducted their research on the Mozilla, Eclipse and GNOME project. The 
    approach can be divided into 5 steps:
      \begin{itemize}
        \item extract and organize bug reports;
        \item preprocess bug reports (i.e. tokenization, stop word removal and stemming);
        \item train the classifier on multiple datasets of bug reports (they used a ratio of
        70\% - training, 30\% - evaluation);
        \item apply the trained classifier on the evaluation set.
      \end{itemize}
        Then, they used precision and recall to evaluate their results.
  \item \textbf{What did they learn?}
    The authors managed to answer all their 4 research questions:
      \begin{itemize}
        \item terms such as deadlock, hand, crash, memory or segfault usually indicate a severe bug;
        however, there are other terms that can indicate the opposite (i.e. non-severe bugs),
        such as typo;
        \item they switched from analysing only the one-line summary to analysing the whole
        textual description as it contains more detailed information, thus the model can be
        trained better; however, when they analysed the results and noticed that they had a large
        number of false positives, which made them switch back to one-line summaries;
        \item the classifier needs a large number of datasets in order to predict better on the
        evaluation set;
        \item it mainly depends on the type of project and how it is structured; thus, in Mozilla
        and GNOME, which share problem-specific characteristics over different components, using a 
        predictor across components sees improvements; however, the dataset needs to be even larger,
        otherwise the performance starts to decrease.
      \end{itemize}
  \item \textbf{Relevance to our work}:
    The authors' findings could help us in better understanding textual description analysis and how
    it can indicate various aspects about a ticket, such as severity. Thus, we can apply the technique
    described in order to differentiate between severe and non-severe bug reports so that we can 
    focus on a specific category in our own research, if we want that.
\end{itemize}

\textbf{Automatic Bug Report Assignment}\cite{anvik2006automating}:
\begin{itemize}
  \item \textbf{What were the goals?}
    The author’s main goal was to determine a way through which bug triaging can be made 
    easier using automatization approaches. 
  \item \textbf{What was the method?}
    The research’s output was an algorithm that can automatically assign bugs to developers based on various types of information fed. The author looked exclusively at ML approaches, examining 8 main types of information (textual description of bug, bug component, OS, hardware, software version, developer who owns the code, current workload of developers, list of devs actively contributing). Moreover, the author considers launching a Firefox extension to test the app even further through human evaluation as well. 
  \item \textbf{What did they learn?}
  The research brings 3 main contributions: methodology for creating semi-automated bug triaging algorithms, a characterization of performance of different approaches, as well as an actual implementation for such a semi-automated big triaget. The author discovered that with further improvements and tweaking, we can reach an even further bug triager.   
  \item \textbf{Relevance to our work}:
  Another project that we can use to learn about bug triaging; also we can contact maybe the author and receive either an executable or the source code for the application. 
\end{itemize}

\textbf{Reducing the effort of bug report triage}\cite{anvik2011reducing}:
\begin{itemize}
  \item \textbf{What were the goals?}
  The authors had one main goal, and that was to find out if costs related to bug triaging can be reduced if ML algorithms were to be introduced. This hypothesis was backed by 2 research questions, which were if the approach would create recommenders that make accurate recommendations and if the humans can make use of the information provided by the approach. 
  \item \textbf{What was the method?}
  They created a bug triager which can present a list of developers from which someone can choose who is the most suitable to fix that specific issue. Also, they improved a previous research the same authors worked on, thus increasing the precision and recall of the algorithm. They employed machine learning techniques to create the algorithm. Then, in order to test it, they implemented a proxy to the actual web service that’s providing the issue repository, thus visualizing results helped conduct the evaluation process.
  \item \textbf{What did they learn?}
  The authors learned that their approach could actually be used in software projects as its accuracy is reasonable. They also learned that the impact of poor software project management, including bug triaging, on software projects can be  quite high.
  \item \textbf{Relevance to our work}:
  The research presented can help us in understanding better if statistical expetiments are helpful in analyzing tickets, as well as give an insight into how important ticket quality and proper triaging are for any software projct. 
\end{itemize}


\textbf{Where should the bugs be fixed?-more accurate information retrieval-based bug localization based on bug reports}\cite{zhou2012should}:
\begin{itemize}
  \item \textbf{What were the goals?}
  The main goal of the paper was to implement a tool that, based on a bug report, could accurately select a number of files where the developer needs to make the necessary changes to fix the issue. They tried to answer 4 main research questions: how many bugs can be successfully lovated by BugLocator, does the revised VSM improve bug localization accuracy, does considering similar bugs improve localization accuracy, can bugLocator outperform other similar methods. 
  \item \textbf{What was the method?}
  The tool firstly performed a textual analysis, looking for similarities between the bug report description and the source code files. Then, it analyzed previous bugs in the repository to find the most similar ones, thus being able to find which files ought to be changed. Lastly, it assigned scores to similar files, the ones with bigger sizes obtaining higher scores as they are more likely to contain bugs. They collected the necessary data from Bugzilla projects and then performed the evaluation. 
  \item \textbf{What did they learn?}
  The buglocator can locate a large percentage of bugs analysing just a small set of source code files. Secondly, the revised VSM outperforms the standard VSM. Moreover, similar bugs can improve the localization accuracy only to a certain extent, while the locator outperformed every other competitor on a multitude of projects. 
  \item \textbf{Relevance to our work}:
  We can try to use BugLocator and even include it in our research as it gives a direct correlation between bug reports and development effort (thus reduced costs). Thus, we can infer even more possible hypotheses regarding our issues’ quality. 
\end{itemize}

\textbf{Analyzing and Relating Bug Report Data for Feature Tracking}\cite{fischer2003analyzing}:
\begin{itemize}
  \item \textbf{What were the goals?}
  The main goal of the paper was to analyze the proximity of software features based on modification and problem report data.
  \item \textbf{What was the method?}
  They employed a method to track features by analyzing and relating bug report data filtered from a release history database. Features are instrumented and tracked, relationships of modification and problem reports to these features are established, and tracked features are visualized to illustrate their otherwise hidden dependencies. 
  \item \textbf{What did they learn?}
  The authors’ approach suggest first to instrument and track features, then establish the relationships of modification and problem reports to these features and lastly visualize the tracked features for illustrating their non apparent dependencies. 
  \item \textbf{Relevance to our work}:
  This paper can give us insight into how bug reports can influence feature implementation throughout the lifetime of the ticket. 
\end{itemize}

\textbf{Duplicate Bug Reports Considered Harmful... Really?}\cite{bettenburg2008duplicate}:
\begin{itemize}
  \item \textbf{What were the goals?}
  The authors wanted to test two main hypotheses, and these were: duplicate bug reports provide developers with information that was not present in the original report and the information in bug duplicates can improve automated triaging techniques. 
  \item \textbf{What was the method?}
  They collected big amounts of data from the Eclipse open source project in XML form. Then, they ran different kinds of textual and statistical analysis on the data to find answers to their research questions. 
  \item \textbf{What did they learn?}
  They reached the conclusion that bug duplicates contain information that is not present in the master reports. This additional data can be helpful for developers and it can also aid automated triaging techniques (e.g. decide who to assign a bug to). 
  \item \textbf{Relevance to our work}:
  The findings presented in this paper could potentially make us consider duplicate tickets in a different way (i.e. do not treat it as disposable but analyze it and see if it maybe adds more information to the original/master report). 
\end{itemize}

\textbf{Summarizing Software Artifacts: A Case Study of Bug Reports}\cite{rastkar2010summarizing}:
\begin{itemize}
  \item \textbf{What were the goals?}
  The main goal of the research was to determine if software artifacts could be summarized effectively and automatically so that developers would need only to analyze summaries instead of full software artifacts (i.e. in our case, bug reports). 
  \item \textbf{What was the method?}
  Firstly, they collected data to analyze from 4 different open source projects (Eclipse, Mozilla, Gnome, KDE). Then, they asked the volunteers (i.e. university students) to annotate the bug reports - write a summary of maximum 250 words in their own sentences. These human-produced annotations were then used by algorithms to learn how to effectively summarize a bug report.  Afterwards, the authors asked the end users of these bug reports, the software developers, to rate the summaries against the original bug reports. 
  \item \textbf{What did they learn?}
  They learned that existing conversation-based extractive summary generators trained on bug reports produce the best results. 
  \item \textbf{Relevance to our work}:
  As the paper revolves around conversations that are attached to a bug (i.e. comments) instead of the actual description of the bug, we can apply the technique described here maybe to generate summaries and analyze them as well in the overall context of ticket quality (maybe together with some sentiment analysis). 
\end{itemize}

\textbf{Improving Bug Triage with Bug Tossing Graphs}\cite{jeong2009improving}:
\begin{itemize}
  \item \textbf{What were the goals?}
  The two main goals of the authors, through their bug tossing graph idea (based on the Markov property), were to discover developer networks and team structures, as well as help to better assign developers to bug reports. 
  \item \textbf{What was the method?}
  They analyzed 145.000 bug reports from Eclipse and 300.000 from Mozilla. Then, using statistical analysis on the bug reports in order to find evolution histories and changes throughout the lifetime of the reports, they created the bug tossing graph. 
  \item \textbf{What did they learn?}
  They learned that it takes a long time to assign and toss bugs. Additionally, they learned that their model reduces tossing steps by up to 72\% and improved the automatic bug assignment by up to 23\%. 
  \item \textbf{Relevance to our work}:
  First paper on importance of tossing and its impact on the software development process. It might prove useful as we shall also inspect tossing and bug assignment in our own research in order to determine ticket quality (i.e. maybe tossing actually reduces the quality of the ticket?). 
\end{itemize}

\textbf{Bug Report Assignee Recommendation using 
Activity Profiles}\cite{Naguib2013BugReportAssignee}:
\begin{itemize}
  \item \textbf{What were the goals?}
    The main goal of the paper is to automatically detect who should be
    the developers who are most suitable to solve a specific issue.
  \item \textbf{What was the method?}
    They employed an algorithm using activity profiles (i.e.\ assign, review,
    resolve activity of the developer) such that, after detecting the prior
    experience, developer's role, and involvement in the project, it could
    recommend who should fix a specific bug. The average accuracy was around
    88\%, much higher than the LDA-SVM technique.
  \item \textbf{What did they learn?}
    They found out that their approach was more accurate than previous work
    done on assignee recommendation, but they think that it might even be
    improved further by combining their technique with the classic LDA-SVM one.
  \item \textbf{Relevance to our work}: this work complements the research
    presented in Anvik et al. \cite{anvik2006should} and it will provide us
    with valuable insight into how automatic triaging should be performed.
\end{itemize}

\textbf{Who should fix this bug?}\cite{anvik2006should}:
\begin{itemize}
  \item \textbf{What were the goals?}
    The authors aimed to create a tool that could automatically assign the 
    bug report to a specific developer based on his/her suitability for that
    specific task.
  \item \textbf{What was the method?}
    They applied a supervised machine learning algorithm on the 
    repositories to learn which developers were best suited for specific tasks,
    thus when a new bug report would come in, a small set of people 
    would be selected. In order to train the algorithm, they looked at
    Bugzilla repositories and selected the free text form of tickets, trying
    to label similar ones based on textual similarities. Once the tickets were 
    labeled and grouped for specific developers, the algorithm would then be
    able to present the triager the set of developers suitable to fix the bug.
  \item \textbf{What did they learn?}
    The most important lesson learned was that collecting data from bug reports 
    and CVS logs was quite challenging. One of the major reasons why they 
    found this aspect hard was that not all CVS comments referenced the 
    specific bug report id.
  \item \textbf{Relevance to our work}
    The paper taught us that bug triaging is hard and that there is almost no 
    automated tool that can choose the perfect developers to work on the task.
    Moreover, the method applied by the authors could prove useful as a 
    learning aid when working with the open source repositories 
    chosen as data sets.
\end{itemize}

\textbf{Software Quality \- The Elusive Target}\cite{kitchenham1996software}:
\begin{itemize}
  \item \textbf{What were the goals?}
    The main goal of the paper is to determine what makes for a good quality
    software project, as well as who are the people in charge of this aspect
    and how should they approach achieving it.
  \item \textbf{What was the method?}
    They tried to define quality in software projects and analyze techniques that 
    measure such quality by looking at other models proposed in different other  
    papers (e.g.\ McCall's quality model, ISO 9126).
  \item \textbf{What did they learn?}
    They learned that quality is very hard to define and there are various 
    factors which need to be taken into consideration, such as the business
    model of the company, the type of the software project (e.g.\ safety
    critical, financial) or the actors which are involved and how they
    coordinate the software activities.
  \item \textbf{Relevance to our work}: this paper is a key paper on software  
    quality and it can prove beneficial in our research by
    giving valuable insights into how software quality can be modeled, thus
    helping us in selecting good quality open source repositories to work with 
    (i.e.\ ticket selection and analysis).
\end{itemize}

\textbf{Code Quality Analysis in OSS}\cite{stamelos2002code}:
\begin{itemize}
  \item \textbf{What were the goals?}
    The article tries to discuss and examine the quality of the source code
    delivered by open source projects.
  \item \textbf{What was the method?}
    They used a set of tools that could automatically inspect various aspects
    of source code. The authors analyzed the 6th release of the OpenSUSE project
    and examined only the components, which are defined by C functions in the
    programs.
  \item \textbf{What did they learn?}
    The research's results show that Linux applications have high quality 
    code standards that one might expect in an open source repository, but 
    the quality is lower than the one implied by the standard. More than half
    of the components were in a high state of quality, but on the other hand, 
    most lower quality components cannot be improved only by applying some
    corrective actions. Thus, even though not all the source code was in 
    an industrial standards shape, there is definitely room for further 
    improvement and open source repositories proved to be of good quality.
  \item \textbf{Relevance to our work}: this work completes the previous paper
    on software quality in general by looking specifically at quality in
    open source projects, which will be our main points for data collection.
\end{itemize}

\textbf{Analysis of Software Cost Estimation}\cite{grimstad2006framework}:
\begin{itemize}
  \item \textbf{What were the goals?}
    The authors are trying to show that poor estimation analysis techniques 
    in software projects will lead to wrong conclusions regarding cost 
    estimation accuracy. Moreover, they also propose a framework for 
    better analysis of software cost estimation error.
  \item \textbf{What was the method?}
    They approached a real-world company where they conducted analysis on
    their cost estimation techniques.
  \item \textbf{What did they learn?}
    They learned that regular, straight-forward types of cost estimation analysis
    techniques error lead them to wrong conclusions. 
  \item \textbf{Relevance to our work}: it showed us that we need to be careful
    when selecting techniques for cost estimation in our own research.
\end{itemize}

\subsection{Measuring Cost and Waste in Software Projects}

\textbf{Secret Life of Bugs}\cite{aranda2009secret}:
\begin{itemize}
\item What were the goals?
  The paper tries to understand and analyze common bug fixing coordination
  activities. Another goal of the paper was to analyze the reliability of
  repositories in terms of software projects coordination and propose
  different directions on how to implement proper tools.
\item What was the method?
  They executed a field study which was split into two parts:
  \begin{itemize}
  \item firstly, they did an exploratory case study of bug repos histories;
  \item secondly, they conducted a survey with professionals (i.e.\ testers,
    developers).
  \end{itemize}
  All data and interviews were conducted using Microsoft bug repositories and
  employees.
\item What did they learn?
  They learned that there are multiple factors which influence the coordination
  activities that revolve around bug fixing, such as organizational, social and
  technical knowledge, thus one cannot infer any conclusions only by automatic
  analysis of the bug repositories. Also, through surverying the professionals,
  they reached the conclusion that there are 8 main goals which can be used for
  better tools and practices:
  \begin{itemize}
  \item probing for ownership;
  \item summit;
  \item probing for expertise;
  \item code review;
  \item triaging;
  \item rapid-fire emailing;
  \item infrequent, direct email;
  \item shotgun emails.
  \end{itemize}
\item Relevance to our work: this paper is one of the key papers for our
  research paper.
\end{itemize}


\textbf{Effects of process maturity on quality, cycle time and effort in software product development}\cite{harter2000effects}:
\begin{itemize}
  \item \textbf{What were the goals?}
  The research tries to investigate the relationship between process maturity, quality, cycle time and effort in software projects. The authors are mainly trying to prove/reject a couple of hypotheses: higher levels of process maturity lead to higher product quality in software projects, higher levels of process maturity are associated with increased cycle time in software products, higher product quality is associated with lower cycle time, higher levels of process maturity lead to increased development efforts in the project and higher product quality is associated with lower development effort.
  \item \textbf{What was the method?}
  In order to test their hypotheses, the authors examined data coming from 30 projects belonging to the systems integration division from a large IT company. The process improvement data was collected via external divisions and by government agencies to provide independent assessments of the firm’s software development processes. Then, they analyzed a couple of key variables and see how they interact: process maturity, product quality, cycle time, development effort, product size, domain/data/decision complexity and requirements ambiguity.
  \item \textbf{What did they learn?}
  They learned that the main features they inspected are additively separable and linear. Moreover, they found out that higher levels of process maturity are associated with significantly higher quality, but also with increased cycle times and development efforts. On the other hand, the reductions in cycle time and effort resulting from improved quality outweigh the marginal increases from achieving more process maturity.
  \item \textbf{Relevance to our work}:
  It can aid our research through better understanding of how various factors can influence the software development processes, thus giving more value to our own research (i.e. being able to automatically determine issue quality could improve better issue writing guidelines/better rules to be enforced, thus reduced triaging and development costs). 
\end{itemize}

\textbf{Waste Identification}\cite{Korkala2014WasteIdentification}:
\begin{itemize}
  \item \textbf{What were the goals?}
    The paper had two main goals: 
    \begin{itemize}
      \item a means to identify communication waste in agile software projects 
        environments;
      \item types of communication waste in agile projects.
    \end{itemize}
  \item \textbf{What was the method?}
    The authors collaborated with a medium-sized American software company
    and conducted a series of observations, informal discussions, documents 
    provided by the organization, as well as semi-structured interviews. Moreover,
    the data collection for waste identification was split into 2 parts:
      \begin{itemize}
        \item \textbf{pre-development}: occured before the actual implementation
          begun (e.g.\ backlog creation);
        \item \textbf{development}: happened throughout the implementation
          process (e.g.\ throughout sprints, retrospectives, sprint reviews,
          communication media).
      \end{itemize}
  \item \textbf{What did they learn?}
    They realized the communication waste can be divided into 5 main categories:
      \begin{itemize}
        \item lack of involvement;
        \item lack of shared understanding;
        \item outdated information;
        \item restricted access to information;
        \item scattered information.
      \end{itemize}
    Also, they learned that their way of identifying these types of waste was
    quite efficient and they even recommend it to companies if they'd like
    to conduct such processes internally.
  \item \textbf{Relevance to our work}: the waste identification process can
    be applied to our work so that we can identify possible causes to poor 
    quality tickets.
\end{itemize}

\textbf{Software Development Waste}\cite{sedano2017software}:
\begin{itemize}
  \item \textbf{What were the goals?}
    The main goal of the paper was to identify main types of waste in software
    development projects.
  \item \textbf{What was the method?}
    They conducted a participant-observation study over a long period of time
    at Pivotal, a consultancy software development company. They also interviewed
    multiple engineers and balanced theoretical sampling with analysis to achieve
    the conclusions.
  \item \textbf{What did they learn?}
    They found out there are nine main types of waste in software projects:
    \begin{itemize}
      \item building the wrong feature or product;
      \item mismanaging backlog;
      \item extraneous cognitive load;
      \item rework;
      \item ineffective communication;
      \item waiting/multitasking;
      \item solutions too complex;
      \item psychological distress.
      \item \textbf{Relevance to our work}: this paper complements the previous 
    one on waste identification\cite{Korkala2014WasteIdentification}.
    \end{itemize}
\end{itemize}

\textbf{Waste in Kanban Projects}\cite{ikonen2010exploring}:
\begin{itemize}
  \item \textbf{What were the goals?}
    The authors are trying to find the main sources of waste in Kanban software
    development projects and categorize/rank them based on severity.
  \item \textbf{What was the method?}
    A controlled case study research was employed in a company called Software
    Factory. They conducted semi-structured interviews with 5 of the team
    members both in the beginning in order to collect data as well as at the end 
    of the whole process to categorize the seven types of waste found. Moreover,
    they also measured the overall success of the project based on Shenar's
    techniques (first-second-third-fourth; project efficiency-imapct on the
    customer-business success-preparing for the future).
  \item \textbf{What did they learn?}
    They reached two main findings:
      \begin{itemize}
        \item they found 7 types of waste throughout the project at various
          development stages:
            \begin{itemize}
              \item partially done work;
              \item extra processes;
              \item extra features;
              \item task switching;
              \item waiting;
              \item motion;
              \item defects.
            \end{itemize}
        \item they reached the conclusion that they couldn't explain the success
          of the project even though waste was found.
      \end{itemize}
  \item \textbf{Relevance to our work}: this work completes the findings 
    from the previous work presented as most of the projects we will work with
    will be Agile, thus Kanban-based in terms of issue management.
\end{itemize}

\subsection{Sentiment Analysis}

\textbf{Thumbs Up or Thumbs Down}\cite{turney2002thumbs}:
\begin{itemize}
  \item \textbf{What were the goals?}
    The main goal of the paper is to detect the overall sentiment transmitted
    through reviews of various types.
  \item \textbf{What was the method?}
    The author created an unsupervised machine learning algorithm that was 
    evaluated on more than 400 reviews on Epinions on various kinds of markets 
    (e.g.\ automobiles, movie). The algorithm implementation was divided 
    into three steps:
      \begin{itemize}
        \item extract phrases containing adjectives or adverbs;
        \item estimate the semantic orientation of the phrases;
        \item classify the review as recommended or not recommended based on
          the semantic orientation calculated at previous step.
      \end{itemize}
  \item \textbf{What did they learn?}
    One thing the author learned that different categories will yield different 
    results. For example, the automobile section on Epinions ranked much higher, 
    84\%, compared to movie reviews, which had an accuracy of 65.83\%. Moreover, 
    most pitfalls of the algorithm could be attributed to multiple factors, such 
    as not using a supervised learning system or limitations of PMI-IR.
  \item \textbf{Relevance to our work}: the method can be applied for extracting
    sentiments from the tickets (description and comments) we will use 
    in our own research.
\end{itemize}

\textbf{Recognizing Contextual Polarity}\cite{wilson2005recognizing}:
\begin{itemize}
  \item \textbf{What were the goals?}
    The paper's main goal is to find efficient ways to distinguish between 
    contextual and prior polarity.
  \item \textbf{What was the method?}
    They used a two step method that used machine learning and a variety of
    features. The first step classified each phrase which had a clue as either
    neutral or polar, followed by taking all phrases marked in the previous step
    and giving them a contextual polarity (e.g.\ positive, negative, 
    both, neutral).
  \item \textbf{What did they learn?}
    Through the method the authors employed, they managed to automatically 
    identify the contextual polarity. As most papers were only looking at the 
    sentiment extracted from the overall document, they managed to get valuable 
    results from looking at specific words and phrases.
  \item \textbf{Relevance to our work}: when analyzing the description and
    comments of the ticket, we can use their method for infering the sentiment 
    transmitted probably more accurately than the technique used in 
    Turney's paper\cite{turney2002thumbs}.
\end{itemize}

\subsection{Conclusion}

%%%%%%%%%%%%%%%%%%%%%%%%%%%%%%%%%%%%%%%%%%%%%%%%%%%%%%%%%%%%%%%%%%%
\section{Proposed Approach}

% 3- 4 pages 

What are we actually going to do?

Research questions, outline experimental design, independent variables, dependent variables.



state how you propose to solve the software development problem. Show that your proposed approach is feasible, but identify any risks.

%%%%%%%%%%%%%%%%%%%%%%%%%%%%%%%%%%%%%%%%%%%%%%%%%%%%%%%%%%%%%%%%%%%
\section{Work Plan}

% 1 - 2 pages  structure as work packages (don't bother with a chart).

show how you plan to organize your work, identifying intermediate deliverables and dates.

%%%%%%%%%%%%%%%%%%%%%%%%%%%%%%%%%%%%%%%%%%%%%%%%%%%%%%%%%%%%%%%%%%%
\bibliographystyle{plainnat}
\bibliography{mprop}
\end{document}
