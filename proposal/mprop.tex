\documentclass{mprop}
\usepackage{graphicx}

% alternative font if you prefer
%\usepackage{times}

% for alternative page numbering use the following package
% and see documentation for commands
%\usepackage{fancyheadings}


% other potentially useful packages
%\uspackage{amssymb,amsmath}
%\usepackage{url}
%\usepackage{fancyvrb}
%\usepackage[final]{pdfpages}

\begin{document}

%%%%%%%%%%%%%%%%%%%%%%%%%%%%%%%%%%%%%%%%%%%%%%%%%%%%%%%%%%%%%%%%%%%
\title{Software Ticket Quality}
\author{Andrei-Mihai Nicolae}
\date{December 18th 2017}
\maketitle
%%%%%%%%%%%%%%%%%%%%%%%%%%%%%%%%%%%%%%%%%%%%%%%%%%%%%%%%%%%%%%%%%%%

%%%%%%%%%%%%%%%%%%%%%%%%%%%%%%%%%%%%%%%%%%%%%%%%%%%%%%%%%%%%%%%%%%%
\tableofcontents
\newpage
%%%%%%%%%%%%%%%%%%%%%%%%%%%%%%%%%%%%%%%%%%%%%%%%%%%%%%%%%%%%%%%%%%%

%%%%%%%%%%%%%%%%%%%%%%%%%%%%%%%%%%%%%%%%%%%%%%%%%%%%%%%%%%%%%%%%%%%
\section{Introduction}\label{intro}

briefly explain the context of the project problem

\subsection{A subsection}
Please note your proposal need not follow the included section headings; this is only a suggested structure. Also add subsections etc as required

%%%%%%%%%%%%%%%%%%%%%%%%%%%%%%%%%%%%%%%%%%%%%%%%%%%%%%%%%%%%%%%%%%%
\section{Statement of Problem}

Clearly state the problem to be addressed in your forthcoming project.
Explain why it would be worthwhile to solve this problem.

%%%%%%%%%%%%%%%%%%%%%%%%%%%%%%%%%%%%%%%%%%%%%%%%%%%%%%%%%%%%%%%%%%%
\section{Background Survey}

\subsection{Introduction}

\subsection{What was covered and why}

\subsection{How papers were found; what terms were searched for + snowballing}

\subsection{Data Quality Metrics}

\subsection{Issue Quality}

\textbf{Secret Life of Bugs}\cite{aranda2009secret}:
\begin{itemize}
\item What were the goals?
  The paper tries to understand and analyze common bug fixing coordination
  activities. Another goal of the paper was to analyze the reliability of
  repositories in terms of software projects coordination and propose
  different directions on how to implement proper tools.
\item What was the method?
  They executed a field study which was split into two parts:
  \begin{itemize}
  \item firstly, they did an exploratory case study of bug repos histories;
  \item secondly, they conducted a survey with professionals (i.e.\ testers,
    developers).
  \end{itemize}
  All data and interviews were conducted using Microsoft bug repositories and
  employees.
\item What did they learn?
  They learned that there are multiple factors which influence the coordination
  activities that revolve around bug fixing, such as organizational, social and
  technical knowledge, thus one cannot infer any conclusions only by automatic
  analysis of the bug repositories. Also, through surverying the professionals,
  they reached the conclusion that there are 8 main goals which can be used for
  better tools and practices:
  \begin{itemize}
  \item probing for ownership;
  \item summit;
  \item probing for expertise;
  \item code review;
  \item triaging;
  \item rapid-fire emailing;
  \item infrequent, direct email;
  \item shotgun emails.
  \end{itemize}
\item Relevance to our work: this paper is one of the key papers for our
  research paper.
\end{itemize}

\textbf{What makes a good bug report}\cite{bettenburg2008makes}:
\begin{itemize}
  \item \textbf{What were the goals?}
  The main goal of the paper was to investigate the \textbf{quality of bug reports} from a developer's
  point of view, based on the typical information found in such a report (e.g.\ stack traces, screenshots).
\item \textbf{What was the method?}
  The authors conducted a massive survey with over 450 respondents. The survey was online and it targeted
  developers from Mozilla, Apache and Eclipse.
\item \textbf{What did they learn?}
  The main conclusion of this research paper was that well written bug reports will more likely get the 
  attention of the developers. Thus, including steps to reproduce the bugs or stack traces proved to increase
  the quality of the bug report. Also, an important achievement reached by the authors was the development of
  a prototype tool called Cuezilla that could estimate, with an accuracy rate of 31\-48\%, the quality of a bug
  report.
\item \textbf{Relevance to our work}: it is relevant to our research in that it provides valuable insight into 
  what makes for a good bug report based on actual professionals' opinions.
\end{itemize}

\textbf{Who should fix this bug?}\cite{anvik2006should}:
\begin{itemize}
  \item \textbf{What were the goals?}
    The authors aimed to create a tool that could automatically assign the 
    bug report to a specific developer based on his/her suitability for that
    specific task.
  \item \textbf{What was the method?}
    They applied a supervised machine learning algorithm on the 
    repositories to learn which developers were best suited for specific tasks,
    thus when a new bug report would come in, a small set of people 
    would be selected. In order to train the algorithm, they looked at
    Bugzilla repositories and selected the free text form of tickets, trying
    to label similar ones based on textual similarities. Once the tickets were 
    labeled and grouped for specific developers, the algorithm would then be
    able to present the triager the set of developers suitable to fix the bug.
  \item \textbf{What did they learn?}
    The most important lesson learned was that collecting data from bug reports 
    and CVS logs was quite challenging. One of the major reasons why they 
    found this aspect hard was that not all CVS comments referenced the 
    specific bug report id.
  \item \textbf{Relevance to our work}
    The paper taught us that bug triaging is hard and that there is almost no 
    automated tool that can choose the perfect developers to work on the task.
    Moreover, the method applied by the authors could prove useful as a 
    learning aid when working with the open source repositories 
    chosen as data sets.
\end{itemize}

\textbf{Software Quality \- The Elusive Target}\cite{kitchenham1996software}:
\begin{itemize}
  \item \textbf{What were the goals?}
    The main goal of the paper is to determine what makes for a good quality
    software project, as well as who are the people in charge of this aspect
    and how should they approach achieving it.
  \item \textbf{What was the method?}
    They tried to define quality in software projects and analyze techniques that 
    measure such quality by looking at other models proposed in different other  
    papers (e.g.\ McCall's quality model, ISO 9126).
  \item \textbf{What did they learn?}
    They learned that quality is very hard to define and there are various 
    factors which need to be taken into consideration, such as the business
    model of the company, the type of the software project (e.g.\ safety
    critical, financial) or the actors which are involved and how they
    coordinate the software activities.
  \item \textbf{Relevance to our work}: this paper is a key paper on software  
    quality and it can prove beneficial in our research by
    giving valuable insights into how software quality can be modeled, thus
    helping us in selecting good quality open source repositories to work with 
    (i.e.\ ticket selection and analysis).
\end{itemize}

\textbf{Code Quality Analysis in OSS}\cite{stamelos2002code}:
\begin{itemize}
  \item \textbf{What were the goals?}
    The article tries to discuss and examine the quality of the source code
    delivered by open source projects.
  \item \textbf{What was the method?}
    They used a set of tools that could automatically inspect various aspects
    of source code. The authors analyzed the 6th release of the OpenSUSE project
    and examined only the components, which are defined by C functions in the
    programs.
  \item \textbf{What did they learn?}
    The research's results show that Linux applications have high quality 
    code standards that one might expect in an open source repository, but 
    the quality is lower than the one implied by the standard. More than half
    of the components were in a high state of quality, but on the other hand, 
    most lower quality components cannot be improved only by applying some
    corrective actions. Thus, even though not all the source code was in 
    an industrial standards shape, there is definitely room for further 
    improvement and open source repositories proved to be of good quality.
  \item \textbf{Relevance to our work}: this work completes the previous paper
    on software quality in general by looking specifically at quality in
    open source projects, which will be our main points for data collection.
\end{itemize}

\textbf{Analysis of Software Cost Estimation}\cite{grimstad2006framework}:
\begin{itemize}
  \item \textbf{What were the goals?}
    The authors are trying to show that poor estimation analysis techniques 
    in software projects will lead to wrong conclusions regarding cost 
    estimation accuracy. Moreover, they also propose a framework for 
    better analysis of software cost estimation error.
  \item \textbf{What was the method?}
    They approached a real-world company where they conducted analysis on
    their cost estimation techniques.
  \item \textbf{What did they learn?}
    They learned that regular, straight-forward types of cost estimation analysis
    techniques error lead them to wrong conclusions. 
  \item \textbf{Relevance to our work}: it showed us that we need to be careful
    when selecting techniques for cost estimation in our own research.
\end{itemize}

\subsection{Measuring Cost and Waste in Software Projects}

\textbf{Waste Identification}\cite{Korkala2014WasteIdentification}:
\begin{itemize}
  \item \textbf{What were the goals?}
    The paper had two main goals: 
    \begin{itemize}
      \item a means to identify communication waste in agile software projects 
        environments;
      \item types of communication waste in agile projects.
    \end{itemize}
  \item \textbf{What was the method?}
    The authors collaborated with a medium-sized American software company
    and conducted a series of observations, informal discussions, documents 
    provided by the organization, as well as semi-structured interviews. Moreover,
    the data collection for waste identification was split into 2 parts:
      \begin{itemize}
        \item \textbf{pre-development}: occured before the actual implementation
          begun (e.g.\ backlog creation);
        \item \textbf{development}: happened throughout the implementation
          process (e.g.\ throughout sprints, retrospectives, sprint reviews,
          communication media).
      \end{itemize}
  \item \textbf{What did they learn?}
    They realized the communication waste can be divided into 5 main categories:
      \begin{itemize}
        \item lack of involvement;
        \item lack of shared understanding;
        \item outdated information;
        \item restricted access to information;
        \item scattered information.
      \end{itemize}
    Also, they learned that their way of identifying these types of waste was
    quite efficient and they even recommend it to companies if they'd like
    to conduct such processes internally.
  \item \textbf{Relevance to our work}: the waste identification process can
    be applied to our work so that we can identify possible causes to poor 
    quality tickets.
\end{itemize}

\textbf{Software Development Waste}\cite{sedano2017software}:
\begin{itemize}
  \item \textbf{What were the goals?}
    The main goal of the paper was to identify main types of waste in software
    development projects.
  \item \textbf{What was the method?}
    They conducted a participant-observation study over a long period of time
    at Pivotal, a consultancy software development company. They also interviewed
    multiple engineers and balanced theoretical sampling with analysis to achieve
    the conclusions.
  \item \textbf{What did they learn?}
    They found out there are nine main types of waste in software projects:
    \begin{itemize}
      \item building the wrong feature or product;
      \item mismanaging backlog;
      \item extraneous cognitive load;
      \item rework;
      \item ineffective communication;
      \item waiting/multitasking;
      \item solutions too complex;
      \item psychological distress.
      \item \textbf{Relevance to our work}: this paper complements the previous 
    one on waste identification\cite{Korkala2014WasteIdentification}.
    \end{itemize}
\end{itemize}

\textbf{Waste in Kanban Projects}\cite{ikonen2010exploring}:
\begin{itemize}
  \item \textbf{What were the goals?}
    The authors are trying to find the main sources of waste in Kanban software
    development projects and categorize/rank them based on severity.
  \item \textbf{What was the method?}
    A controlled case study research was employed in a company called Software
    Factory. They conducted semi-structured interviews with 5 of the team
    members both in the beginning in order to collect data as well as at the end 
    of the whole process to categorize the seven types of waste found. Moreover,
    they also measured the overall success of the project based on Shenar's
    techniques (first-second-third-fourth; project efficiency-imapct on the
    customer-business success-preparing for the future).
  \item \textbf{What did they learn?}
    They reached two main findings:
      \begin{itemize}
        \item they found 7 types of waste throughout the project at various
          development stages:
            \begin{itemize}
              \item partially done work;
              \item extra processes;
              \item extra features;
              \item task switching;
              \item waiting;
              \item motion;
              \item defects.
            \end{itemize}
        \item they reached the conclusion that they couldn't explain the success
          of the project even though waste was found.
      \end{itemize}
  \item \textbf{Relevance to our work}: this work completes the findings 
    from the previous work presented as most of the projects we will work with
    will be Agile, thus Kanban-based in terms of issue management.
\end{itemize}


\subsection{Sentiment Analysis}

% \textbf{Thumbs Up or Thumbs Down}\cite{kitchenham1996software}:
% \begin{itemize}
%   \item \textbf{What were the goals?}
%   \item \textbf{What was the method?}
%   \item \textbf{What did they learn?}
%   \item \textbf{Relevance to our work}:
% \end{itemize}

\textbf{Thumbs Up or Thumbs Down}\cite{turney2002thumbs}:
\begin{itemize}
  \item \textbf{What were the goals?}
    The main goal of the paper is to detect the overall sentiment transmitted
    through reviews of various types.
  \item \textbf{What was the method?}
    The author created an unsupervised machine learning algorithm that was 
    evaluated on more than 400 reviews on Epinions on various kinds of markets 
    (e.g.\ automobiles, movie). The algorithm implementation was divided 
    into three steps:
      \begin{itemize}
        \item extract phrases containing adjectives or adverbs;
        \item estimate the semantic orientation of the phrases;
        \item classify the review as recommended or not recommended based on
          the semantic orientation calculated at previous step.
      \end{itemize}
  \item \textbf{What did they learn?}
    One thing the author learned that different categories will yield different 
    results. For example, the automobile section on Epinions ranked much higher, 
    84\%, compared to movie reviews, which had an accuracy of 65.83\%. Moreover, 
    most pitfalls of the algorithm could be attributed to multiple factors, such 
    as not using a supervised learning system or limitations of PMI-IR.
  \item \textbf{Relevance to our work}: the method can be applied for extracting
    sentiments from the tickets (description and comments) we will use 
    in our own research.
\end{itemize}

\textbf{Recognizing Contextual Polarity}\cite{wilson2005recognizing}:
\begin{itemize}
  \item \textbf{What were the goals?}
    The paper's main goal is to find efficient ways to distinguish between 
    contextual and prior polarity.
  \item \textbf{What was the method?}
    They used a two step method that used machine learning and a variety of
    features. The first step classified each phrase which had a clue as either
    neutral or polar, followed by taking all phrases marked in the previous step
    and giving them a contextual polarity (e.g.\ positive, negative, 
    both, neutral).
  \item \textbf{What did they learn?}
    Through the method the authors employed, they managed to automatically 
    identify the contextual polarity. As most papers were only looking at the 
    sentiment extracted from the overall document, they managed to get valuable 
    results from looking at specific words and phrases.
  \item \textbf{Relevance to our work}: when analyzing the description and
    comments of the ticket, we can use their method for infering the sentiment 
    transmitted probably more accurately than the technique used in 
    Turney's paper\cite{turney2002thumbs}.
\end{itemize}

\subsection{Conclusion}

%%%%%%%%%%%%%%%%%%%%%%%%%%%%%%%%%%%%%%%%%%%%%%%%%%%%%%%%%%%%%%%%%%%
\section{Proposed Approach}

state how you propose to solve the software development problem. Show that your proposed approach is feasible, but identify any risks.

%%%%%%%%%%%%%%%%%%%%%%%%%%%%%%%%%%%%%%%%%%%%%%%%%%%%%%%%%%%%%%%%%%%
\section{Work Plan}

show how you plan to organize your work, identifying intermediate deliverables and dates.

%%%%%%%%%%%%%%%%%%%%%%%%%%%%%%%%%%%%%%%%%%%%%%%%%%%%%%%%%%%%%%%%%%%
\bibliographystyle{plain}
\bibliography{mprop}
\end{document}
