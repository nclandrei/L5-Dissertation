\documentclass{mprop}
\usepackage{graphicx}

% alternative font if you prefer
%\usepackage{times}

% for alternative page numbering use the following package
% and see documentation for commands
%\usepackage{fancyheadings}


% other potentially useful packages
%\uspackage{amssymb,amsmath}
%\usepackage{url}
%\usepackage{fancyvrb}
%\usepackage[final]{pdfpages}

\begin{document}

%%%%%%%%%%%%%%%%%%%%%%%%%%%%%%%%%%%%%%%%%%%%%%%%%%%%%%%%%%%%%%%%%%%
\title{Software Ticket Quality}
\author{Andrei-Mihai Nicolae}
\date{December 18th 2017}
\maketitle
%%%%%%%%%%%%%%%%%%%%%%%%%%%%%%%%%%%%%%%%%%%%%%%%%%%%%%%%%%%%%%%%%%%

%%%%%%%%%%%%%%%%%%%%%%%%%%%%%%%%%%%%%%%%%%%%%%%%%%%%%%%%%%%%%%%%%%%
\tableofcontents
\newpage
%%%%%%%%%%%%%%%%%%%%%%%%%%%%%%%%%%%%%%%%%%%%%%%%%%%%%%%%%%%%%%%%%%%

%%%%%%%%%%%%%%%%%%%%%%%%%%%%%%%%%%%%%%%%%%%%%%%%%%%%%%%%%%%%%%%%%%%
\section{Introduction}\label{intro}

briefly explain the context of the project problem

\subsection{A subsection}
Please note your proposal need not follow the included section headings; this is only a suggested structure. Also add subsections etc as required

%%%%%%%%%%%%%%%%%%%%%%%%%%%%%%%%%%%%%%%%%%%%%%%%%%%%%%%%%%%%%%%%%%%
\section{Statement of Problem}

Clearly state the problem to be addressed in your forthcoming project.
Explain why it would be worthwhile to solve this problem.

%%%%%%%%%%%%%%%%%%%%%%%%%%%%%%%%%%%%%%%%%%%%%%%%%%%%%%%%%%%%%%%%%%%
\section{Background Survey}

\subsection{Introduction}

\subsection{What was covered and why}

\subsection{How papers were found; what terms were searched for + snowballing}

\subsection{Data Quality Metrics}

\subsection{Issue Quality}

Secret Life of Bugs (\citep{aranda2009secret}):
\begin{itemize}
\item What were the goals?
  The paper tries to understand and analyze common bug fixing coordination
  activities. Another goal of the paper was to analyze the reliability of
  repositories in terms of software projects coordination and propose
  different directions on how to implement proper tools.
\item What was the method?
  They executed a field study which was split into two parts:
  \begin{itemize}
  \item firstly, they did an exploratory case study of bug repos histories;
  \item secondly, they conducted a survey with professionals (i.e.\ testers,
    developers).
  \end{itemize}
  All data and interviews were conducted using Microsoft bug repositories and
  employees.
\item What did they learn?
  They learned that there are multiple factors which influence the coordination
  activities that revolve around bug fixing, such as organizational, social and
  technical knowledge, thus one cannot infer any conclusions only by automatic
  analysis of the bug repositories. Also, through surverying the professionals,
  they reached the conclusion that there are 8 main goals which can be used for
  better tools and practices:
  \begin{itemize}
  \item probing for ownership;
  \item summit;
  \item probing for expertise;
  \item code review;
  \item triaging;
  \item rapid-fire emailing;
  \item infrequent, direct email;
  \item shotgun emails.
  \end{itemize}

\item Link to other work
  \begin{itemize}
    \item @inproceedings{anvik2006should,
  title={Who should fix this bug?},
  author={Anvik, John and Hiew, Lyndon and Murphy, Gail C},
  booktitle={Proceedings of the 28th international conference on Software engineering},
  pages={361--370},
  year={2006},
  organization={ACM}
    }
      \item @inproceedings{bettenburg2008makes,
  title={What makes a good bug report?},
  author={Bettenburg, Nicolas and Just, Sascha and Schr{\"o}ter, Adrian and Weiss, Cathrin and Premraj, Rahul and Zimmermann, Thomas},
  booktitle={Proceedings of the 16th ACM SIGSOFT International Symposium on Foundations of software engineering},
  pages={308--318},
  year={2008},
  organization={ACM}
      }
        \item @inproceedings{ko2007information,
  title={Information needs in collocated software development teams},
  author={Ko, Andrew J and DeLine, Robert and Venolia, Gina},
  booktitle={Software Engineering, 2007. ICSE 2007. 29th International Conference on},
  pages={344--353},
  year={2007},
  organization={IEEE}
        }
  \end{itemize}
\item Relevance to our work: this paper is one of the key papers for our
  research paper.
\end{itemize}

\citep{Korkala2014WasteIdentification}
\begin{itemize}
\item What were the goals?
  
\item What was the method?
\item What did they learn?
\item Link to other work
\item Relevance to our work
\end{itemize}

\citep{aranda2009secret}
\begin{itemize}
\item What were the goals?
\item What was the method?
\item What did they learn?
\item Link to other work
\item Relevance to our work
\end{itemize}

\citep{aranda2009secret}
\begin{itemize}
\item What were the goals?
\item What was the method?
\item What did they learn?
\item Link to other work
\item Relevance to our work
\end{itemize}

\citep{aranda2009secret}
\begin{itemize}
\item What were the goals?
\item What was the method?
\item What did they learn?
\item Link to other work
\item Relevance to our work
\end{itemize}

\subsection{Measuring Cost and Waste in Software Projects}

\subsection{Sentiment Analysis}

\subsection{Conclusion}

%%%%%%%%%%%%%%%%%%%%%%%%%%%%%%%%%%%%%%%%%%%%%%%%%%%%%%%%%%%%%%%%%%%
\section{Proposed Approach}

state how you propose to solve the software development problem. Show that your proposed approach is feasible, but identify any risks.

%%%%%%%%%%%%%%%%%%%%%%%%%%%%%%%%%%%%%%%%%%%%%%%%%%%%%%%%%%%%%%%%%%%
\section{Work Plan}

show how you plan to organize your work, identifying intermediate deliverables and dates.

%%%%%%%%%%%%%%%%%%%%%%%%%%%%%%%%%%%%%%%%%%%%%%%%%%%%%%%%%%%%%%%%%%%
\bibliographystyle{plain}
\bibliography{mprop}
\end{document}
